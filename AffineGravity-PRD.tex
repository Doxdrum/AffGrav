\documentclass[twocolumn,aps,
  showpacs,showkeys,prd,superscriptaddress]{revtex4-1}

\usepackage{amsmath,amsthm,latexsym,amssymb,amsfonts,epsfig}
\usepackage{xcolor}
\usepackage{ulem}
%\usepackage{authblk}
\usepackage[%
  colorlinks=true,
  urlcolor=blue,
  linkcolor=blue,
  citecolor=blue
]{hyperref}
\usepackage{etoolbox}
\usepackage{breqn}

\makeatletter
\let\cat@comma@active\@empty
\makeatother

%------------------
%--------- Definitions
%------------------
\newcommand\md{{\mathrm{d}}}
%\newcommand\mx{{\mathrm{x}}}
\newcommand\my{{\mathrm{y}}}
\newcommand\mz{{\mathrm{z}}}
\renewcommand\l{\left}
\renewcommand\r{\right}
\renewcommand\ll{\left\{}
\newcommand\rl{\right\}}
\newcommand\lp{\left(}
\newcommand\rp{\right)}
\newcommand\lc{\left[}
\newcommand\rc{\right]}
\newcommand\la{\left\langle}
\newcommand\ra{\right\rangle} 
\newcommand{\hl}[1]{{\color{red} \textbf{#1}}}


\usepackage{amsmath,amssymb,amsfonts,dsfont,mathrsfs,amsthm}
\usepackage{graphicx}
\usepackage{centernot}
\usepackage{hyperref}
\usepackage{xcolor}
%% \usepackage{comment}
\hypersetup{linktocpage,colorlinks=true,urlcolor=blue!80!red,linkcolor=blue,citecolor=red}
%% \usepackage{feynmf}
\usepackage{siunitx}
\usepackage{array}
\usepackage{ulem}
%% \usepackage{tikz}
\usepackage{braket}
%% \usetikzlibrary{shapes,
%%   snakes,
%%   decorations.pathmorphing,
%%   decorations.markings,
%%   calc,
%%   shadows.blur,
%%   shadings}
%% \usepackage[framemethod=tikz]{mdframed}

%% \makeatletter
%% % gluon decoration (based on the original coil decoration)
%% \pgfdeclaredecoration{gluon}{coil}
%% {
%%   \state{coil}[switch if less than=%
%%     0.5\pgfdecorationsegmentlength+%>
%%     \pgfdecorationsegmentaspect\pgfdecorationsegmentamplitude+%
%%     \pgfdecorationsegmentaspect\pgfdecorationsegmentamplitude to last,
%%                width=+\pgfdecorationsegmentlength]
%%   {
%%     \pgfpathcurveto
%%     {\pgfpoint@oncoil{0    }{ 0.555}{1}}
%%     {\pgfpoint@oncoil{0.445}{ 1    }{2}}
%%     {\pgfpoint@oncoil{1    }{ 1    }{3}}
%%     \pgfpathcurveto
%%     {\pgfpoint@oncoil{1.555}{ 1    }{4}}
%%     {\pgfpoint@oncoil{2    }{ 0.555}{5}}
%%     {\pgfpoint@oncoil{2    }{ 0    }{6}}
%%     \pgfpathcurveto
%%     {\pgfpoint@oncoil{2    }{-0.555}{7}}
%%     {\pgfpoint@oncoil{1.555}{-1    }{8}}
%%     {\pgfpoint@oncoil{1    }{-1    }{9}}
%%     \pgfpathcurveto
%%     {\pgfpoint@oncoil{0.445}{-1    }{10}}
%%     {\pgfpoint@oncoil{0    }{-0.555}{11}}
%%     {\pgfpoint@oncoil{0    }{ 0    }{12}}
%%   }
%%   \state{last}[next state=final]
%%   {
%%     \pgfpathcurveto
%%     {\pgfpoint@oncoil{0    }{ 0.555}{1}}
%%     {\pgfpoint@oncoil{0.445}{ 1    }{2}}
%%     {\pgfpoint@oncoil{1    }{ 1    }{3}}
%%     \pgfpathcurveto
%%     {\pgfpoint@oncoil{1.555}{ 1    }{4}}
%%     {\pgfpoint@oncoil{2    }{ 0.555}{5}}
%%     {\pgfpoint@oncoil{2    }{ 0    }{6}}
%%   }
%%   \state{final}{}
%% }

%% \def\pgfpoint@oncoil#1#2#3{%
%%   \pgf@x=#1\pgfdecorationsegmentamplitude%
%%   \pgf@x=\pgfdecorationsegmentaspect\pgf@x%
%%   \pgf@y=#2\pgfdecorationsegmentamplitude%
%%   \pgf@xa=0.083333333333\pgfdecorationsegmentlength%
%%   \advance\pgf@x by#3\pgf@xa%
%% }
%% \makeatother

%% \tikzset{
%%   boson/.style={decorate,decoration={gluon,segment length=9pt,aspect=0}},
%%   % style to apply some styles to each segment of a path
%%   on each segment/.style={
%%     decorate,
%%     decoration={
%%       show path construction,
%%       moveto code={},
%%       lineto code={
%%         \path [#1]
%%         (\tikzinputsegmentfirst) -- (\tikzinputsegmentlast);
%%       },
%%       curveto code={
%%         \path [#1] (\tikzinputsegmentfirst)
%%         .. controls
%%         (\tikzinputsegmentsupporta) and (\tikzinputsegmentsupportb)
%%         ..
%%         (\tikzinputsegmentlast);
%%       },
%%       closepath code={
%%         \path [#1]
%%         (\tikzinputsegmentfirst) -- (\tikzinputsegmentlast);
%%       },
%%     },
%%   },
%%   % style to add an arrow in the middle of a path
%%   mid arrow/.style={postaction={decorate,decoration={
%%         markings,
%%         mark=at position .5 with {\arrow[#1]{stealth}}
%%       }}},
%% }


%-------------------------------Theorems
\newtheorem{Def}{Definition}
\newtheorem{Thm}{Theorem}
\newtheorem{Lem}{Lemma}
\newtheorem{Pos}{Postulate}
\newtheorem{Exa}{Example}
\newtheorem{Cor}{Corrolary}
\newtheorem{Pro}{Proposition}

%---------------------------------New commands
\newcommand{\A}{\mathcal{A}} 
\newcommand{\abs}[1]{\left|{#1}\right|}
\newcommand{\Ag}{\mathcal{A}_{(1)}}
\newcommand{\Agf}{\boldsymbol{\mathcal{A}}}
\newcommand{\Af}{\, {\mathbf{A}} }
\newcommand{\AF}[1]{\, {\mathbf{A}}_{({#1})} }
\newcommand{\hAf}{\, \hat{\mathbf{A}}_{(1)} }
\newcommand{\hAF}[1]{\, \hat{\mathbf{A}}_{(#1)} }
\newcommand{\bboxed}[1]{{\color{red}{\boxed{\boxed{\textcolor{black}{#1}}}}}}
\newcommand{\C}{\mathbb{C}}
\newcommand{\Cl}{\mathcal{C}\!\ell}
\newcommand{\cdf}[1][]{\,{\boldsymbol{\mathcal{D}}}{#1}\!}
\newcommand{\covd}{\mathcal{D}}
\newcommand{\D}{\mathscr{D}}
\newcommand{\df}[1][]{\,{\mathbf{d}}{#1}\!}
\newcommand{\dfd}{\,{\mathbf{d}}^\dag\!}
\newcommand{\ele}[2][]{\frac{d}{dt}\left(\frac{\partial\mathcal{L}}{\partial \dot{#2}^{#1}}\right) - \frac{\partial\mathcal{L}}{\partial {#2}^{#1}}}
\newcommand{\fele}[2][]{\partial_\mu \left(\frac{\delta\mathcal{L}}{\delta\left(\partial_\mu{#2}^{#1}\right)}\right) - \frac{\delta\mathcal{L}}{\delta {#2}^{#1}}}
\newcommand{\vb}[1]{\vec{e}_{#1}}
%% \newcommand{\fb}[1]{\widetilde{e}{}^{\, #1}}
\newcommand{\fb}[1]{\widetilde{e}{}^{\, #1}}
\newcommand\fder[3][]{\frac{\delta^{#1}{#2}}{\delta {#3}^{#1}}}
\newcommand\fdern[4][]{\frac{\delta^{#1}{#2}}{\delta {#3} \cdots \delta {#4}}}
\newcommand{\F}{\,\boldsymbol{\mathcal{F}}}
\newcommand{\Fg}{\,\boldsymbol{\mathcal{F}}_{(2)}}
\newcommand{\Ff}{\,{\mathbf{F}}}
\newcommand{\FF}[1]{\,{\mathbf{F}}_{(#1)}}
\newcommand{\hFF}[1]{\,{\mathbf{\hat{F}}}_{(#1)}}
\newcommand{\fy}{\centernot}
\newcommand{\G}{\mathscr{G}}
\newcommand{\ga}{\gamma}
\newcommand{\gf}{\,\boldsymbol{\gamma}}
\newcommand{\Ga}{\Gamma}
\newcommand{\conn}[3]{\left(\Gamma_{#1}\right)^{#2}{}_{#3}}
\newcommand{\Ha}{\mathscr{H}}
\newcommand{\He}{\mathbb{H}}
\newcommand{\Hi}{\mathcal{H}}
\newcommand{\Hint}{\underline{\sc Hint:} }
\DeclareMathOperator{\hs}{\,\star\!\!}
\newcommand{\J}{\mathscr{J}}
\newcommand{\K}{\mathbb{K}}
\newcommand{\KK}{Kaluza-Klein{\;}}
\newcommand{\Lag}{\mathscr{L}}
\newcommand{\Li}{\mathcal{L}}
\newcommand{\La}[1][]{\triangle_{#1}}
\newcommand{\Lap}{\nabla^2}
\newcommand{\lr}[1]{\stackrel{\leftrightarrow}{#1}}
\newcommand{\M}{\ensuremath{\mathscr{M}}}
\newcommand{\Mi}{\mathcal{M}}
\newcommand{\MN}{Maldacena-N\'u\~nez{\;}}
\newcommand{\N}{\ensuremath{\mathscr{N}}}
\newcommand{\Na}{\mathbb{N}}
\newcommand{\No}{\mathcal{N}}
\newcommand{\norm}[1]{\left\|#1\right\|}
\newcommand{\Op}{\mathcal{O}}
\newcommand{\Or}{\mathscr{O}}
\newcommand\pder[3][]{\frac{\partial^{#1}{#2}}{\partial {#3}^{#1}}}
\newcommand\pdern[4][]{\frac{\partial^{#1}#2}{\partial #3\cdots\partial #4}}
\newcommand{\Qh}[1][]{\ensuremath{\hat{Q}_{#1}}}
\newcommand{\R}{\mathbb{R}}
\newcommand{\Ri}{\mathcal{R}}
%% \newcommand{\S}{\mathscr{S}}
\newcommand{\SM}{Standard Model {}}%\mathscr{S}}
\newcommand{\T}{\mathscr{T}}
\newcommand{\tor}{\mathcal{T}}
\newcommand{\tors}[3]{\mathcal{T}{}_{#1}{}^{#2}{}_{#3}}
\newcommand\vdj[1]{\left< \Delta J \right>_{#1}}
\newcommand{\w}{{\scriptstyle\wedge}\!}
\newcommand{\Z}{\mathbb{Z}}

\newcommand{\dbar}[1]{\ensuremath{\mathchar'26\mkern-12mu \mathrm{d}^{#1}}\!}


%--------------------------- New Greek
\newcommand{\tht}{\ensuremath{\theta}}
\newcommand{\bet}{\ensuremath{\bar{\eta}}}
\newcommand{\bps}{\ensuremath{\bar{\psi}}}
\newcommand{\bc}{\ensuremath{\bar{\chi}}}
\newcommand{\Bps}{\ensuremath{\bar{\Psi}}}
\newcommand{\Bx}{\ensuremath{\bar{\Xi}}}
\newcommand{\bph}{\ensuremath{\bar{\phi}}}
\newcommand{\vph}{\ensuremath{\varphi}}
\newcommand{\bvph}{\ensuremath{\bar{\varphi}}}
\newcommand{\bth}{\ensuremath{\bar{\theta}}}
\newcommand{\hph}{\ensuremath{\hat{\phi}}}

\newcommand{\bs}[1]{\boldsymbol{#1}}


\renewcommand{\div}{{\mathbf{div}}}
\newcommand{\grad}{{\mathbf{grad}}}
\newcommand{\curl}{{\mathbf{curl}}}

\newcommand\VI[2]{\,\hat{e}^{\hat{#1}}_{\hat{#2}}}
\newcommand\VIF[1]{\,\hat{\boldsymbol{e}}^{\hat{#1}}}
\newcommand\VIN[2]{\;\hat{E}^{\hat{#1}}_{\hat{#2}}}
\newcommand\VINF[1]{\,\hat{\mathbf{E}}_{\hat{#1}}}
\newcommand\hvi[2]{\;\hat{e}^{{#1}}_{{#2}}}
\newcommand\hvin[2]{\;\hat{E}^{{#1}}_{{#2}}}
\newcommand\hvif[1]{\,\hat{\mathbf{e}}^{{#1}}}
\newcommand\hvinf[1]{\,\hat{\mathbf{E}}_{{#1}}}
\newcommand\vi[2]{e^{{#1}}_{{#2}}}
\newcommand\vin[2]{E^{{#1}}_{{#2}}}
\newcommand\vif[1]{\,{\mathbf{e}}^{{#1}}}
\newcommand\vinf[1]{\,{\mathbf{E}}_{{#1}}}
%% \newcommand\Vi[2]{e^{\hat{#1}}_{\hat{#2}}}
%% \newcommand\Vin[2]{E^{\hat{#1}}_{\hat{#2}}}
\newcommand\GAM[1]{{\gamma}^{\hat{#1}}}
%% \newcommand\Gam[1]{\gamma^{\hat{#1}}} 
\newcommand\gam[1]{\gamma^{{#1}}}
\newcommand\hgam[1]{\hat{\gamma}^{{#1}}}
\newcommand\NAB[1]{\hat{\nabla}_{\hat{#1}}}
\newcommand\Nab[1]{\nabla_{\hat{#1}}}
\newcommand\nab[1]{\nabla_{{#1}}}
\newcommand\PA[1]{\partial_{\hat{#1}}}
\newcommand\pa[1]{\partial_{{#1}}}
\newcommand\PAU[1]{\partial^{\hat{#1}}}
\newcommand\pau[1]{\partial^{{#1}}}
\newcommand\lf[1]{{\omega}^{{#1}}}
\newcommand\lft[1]{\hat{\omega}^{{#1}}}
\newcommand\SPI[1]{\;\hat{\omega}_{\hat{#1}}}
\newcommand\SPIF[2]{\,\hat{\boldsymbol{\omega}}^{\hat{#1}}{}_{\hat{#2}}}
%% \newcommand\Spi[1]{\omega_{\hat{#1}}}
\newcommand\spi[1]{\omega_{{#1}}}
\newcommand\tspi[1]{\tilde{\omega}_{{#1}}}
\newcommand\spif[2]{\,{\boldsymbol{\omega}}^{{#1}}{}_{{#2}}}
\newcommand\tspif[2]{\,{\tilde{\boldsymbol{\omega}}}^{{#1}}{}_{{#2}}}
\newcommand\hspi[1]{\hat{\omega}_{{#1}}}
\newcommand\hspif[2]{\,\hat{\boldsymbol{\omega}}^{{#1}}{}_{{#2}}}
%%%%%%%%% Beware of the inconsistency between
%%%%%%%%% \Rif and \RIF
\newcommand{\RIF}[2]{\,\hat{\boldsymbol{\mathcal{R}}}^{\hat{#1}}{}_{\hat{#2}}}
\newcommand{\hRif}[2]{\,\hat{\boldsymbol{\mathcal{R}}}^{{#1}}{}_{{#2}}}
\newcommand{\Rif}[2]{\,\boldsymbol{\mathcal{R}}^{{#1}}{}_{{#2}}}
\newcommand{\tRif}[2]{\,\tilde{\boldsymbol{\mathcal{R}}}^{{#1}}{}_{{#2}}}
\newcommand{\Tf}[1]{\,\boldsymbol{\mathcal{T}}^{#1}}
\newcommand{\TF}[1]{\,\hat{\boldsymbol{\mathcal{T}}}^{\hat{#1}}}
\newcommand{\Tor}[2]{\mathcal{T}^{#1}{}_{#2}}
\newcommand{\cont}[3]{\mathcal{K}_{#1}{}^{#2}{}_{#3}}
\newcommand{\contf}[2]{\,\boldsymbol{\mathcal{K}}^{#1}{}_{#2}}
\newcommand{\hcont}[3]{\hat{\mathcal{K}}_{#1}{}^{#2}{}_{#3}}
\newcommand{\hcontf}[2]{\,\hat{\boldsymbol{\mathcal{K}}}^{#1}{}_{#2}}
\newcommand{\CONT}[3]{\hat{\mathcal{K}}_{\hat{#1}}{}^{\hat{#2}}{}_{\hat{#3}}}
\newcommand{\CONTF}[2]{\,\hat{\boldsymbol{\mathcal{K}}}^{\hat{#1}}{}_{\hat{#2}}}

%% \newcommand{\ket}[1]{\left.\left|#1\right.\right>}
%% \newcommand{\bra}[1]{\left.\left<#1\right.\right|}
\renewcommand\bra[1]{\Bra{#1}}
\renewcommand\ket[1]{\Ket{#1}}
\newcommand{\bkt}[3]{\Braket{ {#1} | {#2} | {#3} } }
\newcommand{\bk}[2]{\Braket{ {#1} | {#2} } }
\newcommand{\comm}[2]{\left[#1,#2\right]}
\newcommand{\acomm}[2]{\left\{#1,#2\right\}}
\newcommand{\vev}[1]{\ensuremath{\left<#1\right>}}
\renewcommand{\set}[1]{\ensuremath{\Set{ #1 }}}

\newcommand{\relphantom}[1]{\mathrel{\phantom{#1}}}

\newcommand{\Ric}{\operatorname{Ric}}
\newcommand*{\diag}{\operatorname{diag}}
\newcommand{\id}{\operatorname{id}}
\newcommand{\tr}{\operatorname{tr}}
\newcommand{\Tr}{\operatorname{Tr}}
\newcommand{\Ker}{\operatorname{Ker}}
\renewcommand{\Im}{\operatorname{Im}}
\newcommand{\sgn}{\operatorname{sgn}}
\newcommand{\Ln}{\operatorname{Ln}}
\newcommand{\Ei}{\operatorname{Ei}}
\newcommand{\csch}{\operatorname{csch}}
\newcommand{\arcsinh}{\operatorname{arcsinh}}
\DeclareMathOperator\Br{Br}

\newcommand{\beq}{\begin{equation}}
\newcommand{\eeq}{\end{equation}}
\newcommand{\ber}{\begin{eqnarray}}
\newcommand{\eer}{\end{eqnarray}}

\renewcommand{\(}{\left(}
\renewcommand{\)}{\right)}
\renewcommand{\[}{\left[}
\renewcommand{\]}{\right]}

\newcommand{\uf}[2][]{\ensuremath{u_{#1}\(\vec{#2}\)}}
\newcommand{\ufb}[2][]{\ensuremath{\bar{u}_{#1}\(\vec{#2}\)}}
\newcommand{\vf}[2][]{\ensuremath{v_{#1}\(\vec{#2}\)}}
\newcommand{\vfb}[2][]{\ensuremath{\bar{v}_{#1}\(\vec{#2}\)}}
\newcommand\pol[2][]{\ensuremath{\varepsilon_{#1}(\vec{#2})}}
\newcommand\polc[2][]{\ensuremath{\varepsilon^*_{#1}(\vec{#2})}}
\newcommand{\ann}[3]{\ensuremath{#1\(\vec{#2},#3\)}}
\newcommand{\cre}[3]{\ensuremath{#1^\dag\(\vec{#2},#3\)}}
%% \newcommand{\uf}[2]{\ensuremath{u\(\vec{#1},#2\)}}
%% \newcommand{\ufb}[2]{\ensuremath{\bar{u}\(\vec{#1},#2\)}}
%% \newcommand{\vf}[2]{\ensuremath{v\(\vec{#1},#2\)}}
%% \newcommand{\vfb}[2]{\ensuremath{\bar{v}\(\vec{#1},#2\)}}
%% \newcommand{\ann}[3]{\ensuremath{#1\(\vec{#2},#3\)}}
%% \newcommand{\cre}[3]{\ensuremath{#1^\dag\(\vec{#2},#3\)}}

\newcommand{\dif}{{\mathrm{d}}}
\newcommand{\difn}[1]{{\mathrm{d}}^{#1}}
\newcommand{\dn}[2]{\,{\mathrm{d}}^{#1}\!{#2}\;}
\newcommand{\dbarn}[2]{\,\dbar{#1}{#2}\;}
\newcommand*{\de}[1]{\mathop{\mathrm{d}#1}\nolimits}% differential, facultative argoment between square parentheses
\newcommand*{\desec}[1][]{\mathop{\mathrm{d^2}#1}\nolimits}% second differential, facultative argoment between square parentheses
\newcommand{\der}[2]{\frac{\de{#1}}{\de{#2}}}% first derivative 
%\newcommand{\pder}[2]{\frac{\pa{}#1}{\pa{}{#2}}}% first derivative 
\newcommand{\inlineder}[2]{\mathrm{d}{#1}/\mathrm{d}{#2}}% in-line first derivative
\newcommand{\dersec}[2]{\frac{{\desec[#1]}}{\de[{#2}^2]}}% second derivative
%% \newcommand{\dx}{\de[x]}% frequently used differentials
%% \newcommand{\dy}{\de[y]}
%\newcommand{\df}{\de[f]}

%------------------
%--------- Format
%------------------
\newcommand{\out}[1]{{\color{red} \sout{#1}}}
\newcommand{\pro}[1]{{\color{blue!70!black} #1}}



\hypersetup{%
  pdftitle={Purely Affine Gravity},
  pdfauthor={Aureliano Skirzewski,}{Oscar Castillo-Felisola},
  pdfkeywords={Affine Gravity,} {Torsion,} {Generalised Gravity.},
  pdflang={English}
}



%------------------
%--------- Document
%------------------
\begin{document}

\title{Purely affine gravity}


\author{Aureliano \surname{Skirzewski}}
\email[Corresponding Author: ]{askirz@gmail.com}
\affiliation{Centro de F\'\i sica Fundamental,  Universidad de los Andes, 5101 M\'erida, Venezuela.}

\author{Oscar \surname{Castillo-Felisola}}
\email{o.castillo.felisola@gmail.com}
\affiliation{Centro Cient\'\i fico Tecnol\'ogico de Valpara\'\i so, Chile.}
\affiliation{Departamento de F\'\i sica, Universidad T\'{e}cnica Federico Santa Mar\'\i a, Casilla 110-V, Valpara\'\i so, Chile.}

%--------- Abstract
\begin{abstract}
  We develop a topological theory of gravity with torsion where metric has a dynamical rather than a kinematical origin. This approach towards gravity resembles pre-geometrical approaches in which a fundamental metric does not exist (see Ref.~\cite{WheelerPre,*Gibbs:1995gj}), but the  affine connection gives place to a local inertial structure. Such feature reminds us of Mach's principle, that assumes the inertial forces, should have dynamical origin. Additionally, a Newtonian gravitational force is obtained in the non-relativistic limit of the theory.
\end{abstract}

\pacs{04.50.-h,04.62.+v,11.25.Mj,11.25.-w}
\keywords{Affine Gravity, Torsion, Generalised Gravity.}


\maketitle


\section{Introduction}

In a critique to Newtonian mechanics, Mach proposed that inertial forces should have a dynamical rather than a kinematical origin. Consider for a second the scenario where there were no stars to be seen and no reference with respect to which an astronaut on empty space, could say that he is rotating. Which reference frame should we take to be inertial for the astronaut?  Thus, the question arises, Should inertial forces  be in correspondence with the presence of matter elsewhere in the universe? For a deeper discussion on the subject of Mach's principle see Ref.~\cite{Lichtenegger:2004re} and references therein.  We wish to call the attention of the reader to the fact that any locally Minkowskian metric in the kinematics of the description of spacetime will introduce a notion of inertial forces at a microscopic level~\cite{Sciama:1964wt}. With this in mind, we will explore the dynamical origin of inertial forces, studying the appearance of a spacetime metric from a renormalizable model to describe the dynamics of the affine connection of a manifold with torsion. In this model, we use the most general action that might be power counting renormalizable that includes only the sixty-four components gauge connection associated with diffeomorphisms invariance.

During the last years there have been an increasing amount of alternative theories of gravity being tested. Yet, General Relativity (GR) has proven to be the most successful theory of gravity.  Still,  it is not as successful as we may wish~\cite{Kiefer:2013jqa}. Part of the problem is that the standard quantization procedure cannot be applied  properly on GR. Moreover, not only it is not renormalizable, but there are  problems with the choice of variables to be quantized and the choice of the Hilbert space to be used. Although we dare not to say anything against metric spacetimes, to sum over all possible field configurations of the metric seems to be wrong, as this would imply summing Euclidean and Minkowski like contributions to the transition amplitudes on equal terms. Additionally, we might also consider the difficulties of  quantizing  non polynomial field theories, and more specifically square roots of the metric that appears in the Hamiltonian in an ADM formulation of GR, such as the square root of the metric's determinant or the Ricci scalar.

In order to bypass some of these issues, several approaches have been designed that use the connection as a fundamental field. For instance, in the context of Cartan formulations of gravity, letting the metric background become flat we can rewrite Einstein-Hilbert's Lagrangian as a function of the torsion field. This approach is known as Teleparallel Gravity and is equivalent to GR in spite of taking torsion as the fundamental gravitational field~(see Ref.~\cite{Teleparallel,Baez:2012bn} and references within).

Furthermore, another alternative description of GR developed initially by Ashtekar uses the spin connection as the fundamental field and the frame field turns out to be its canonically conjugated momentum. In the context of Loop Quantum Gravity (LQG), using Ashtekar connection, a successful quantization program has been achieved~\cite{Ashtekar:2004eh,thiemann2007loop}. Originally, this approach towards quantum gravity addressed the concerns of the quantization of non polynomial functions of the gravitational field but later on, it turned out that diffeomorphisms symmetry would not show up when the quantum operators were not of the correct density weight, which forces one to reintroduce the squared root~\cite{Thiemann:1996aw}.  Some  strengths of this quantization program lies within a theorem by Lewandowski \emph{et al.} in Ref.~\cite{Lewandowski:2005jk} that states the only diffeomorphisms invariant Hilbert space that supports the Heisenberg algebra, for the connection and its associated momentum, is the one of Loop Quantum Gravity. In spite of its success, LQG has not advanced enough to conclude that its low energy effective description is GR. Currently, there is no clue about the LQG effective description at other scales,  nor its continuum spacetime limit either. Therefore, we cannot conclude that the search for a fundamental theory of gravitational interactions has ended. On the contrary, there are increasingly many alternatives to the usual metric description of Gravity and they all must be tested against experiments and observations.

In this article we study a power counting renormalizable,  diffeomorphism invariant model  consisting  solely of an affine (linear) connection  (with torsion). We expect  this model may  overcome  the uniqueness theorem about diffeomorphism invariant theories of connections, since we  have no fundamental metric field to quantize. The earliest model that argues  a description of gravitational interaction in terms of connections as fundamental fields  was presented by Eddington~\cite{Eddington1923math}, for an spacetime with positive cosmological constant. He proposed the square root of the determinant of the Ricci tensor as the gravitational Lagrangian. His aim was not to solve the problems of quantum gravity,  others have emphasized the character of GR as a gauge theory in order to address the issues of quantization and regularization, as in LQG. Yet others like N. Poplawski~\cite{Poplawski:2012bw} and K. Krasnov~\cite{Krasnov:2011pp} have advanced the road towards a pure connection gravity theory.

The aim of this work is to describe certain aspects of the gravitational interaction in four dimensions. To this end we have studied a  symmetric tensor density which acts like an ``inverse metric'' can be obtained by identifying (see Ref.~\cite{Poplawski:2012bw})
\begin{equation}\label{metric}
  \sqrt{g} g^{\mu\nu} = \frac{\delta\ }{\delta R_{(\mu\nu)}} S[\Gamma]
\end{equation}
where $g$ is the inverse of the determinant of $g^{\mu\nu}$.

The article is organized as follows: In Sec.~\ref{sec:3} we analyse the more general ``gravitational'' theory constructed with the affine connection and power counting renormalizable. In Sec.~\ref{sec:4} we study the four-dimensional model, constructed under the same precepts than before. Additionally, we found solutions to the equations of motion assuming static and homogeneous background, and show that in the non-relativistic limit of the theory the gravitational potential is Newtonian. Finally, in Sec.~\ref{sec:dis} we briefly discuss the reaching consequences of the constructed model.



%--------- Scattered Ideas
\section{\label{sec:3} Warming up: The three-dimensional case}


Formally, the curvature of a manifold is defined through the commutator of covariant derivatives under diffeomorphims, $\nabla_\mu$, but for general choice of the connection, $\Ga^\mu{}_{\nu\lambda}$, there is an extra contribution given by its antisymmetric part in the lower indices, $T^\mu{}_{\nu\lambda} = 2\Ga^\mu{}_{[\nu\lambda]}$. Therefore, the commutator of the covariant derivatives yields,
\begin{equation}
  \comm{\nab{\mu}}{\nab{\nu}}V^\rho = R_{\mu\nu}{}^\rho{}_\lambda V^\lambda - T^\rho{}_{\mu\nu}\nab{\rho}V^\rho.
  \label{curvdef}
\end{equation}
Note that $T^\rho{}_{\mu\nu}$ is a $\tfrac{d^2(d-1)}{2}$ dimensional tensor representation under diffeomorphisms.

In order to build topological invariants of density one, we can use the skew-symmetric Levi-\v{C}ivita tensor $\epsilon^{\mu_1\mu_2\dots\mu_n}$ in $n$-dimensional space(-time).

\begin{widetext}
  With these ingredients, in a three-dimensional space we write an action 
  \begin{dmath}
    \label{accion3d}
    %\begin{split}      
    S[\Gamma] =
    \int \dn{3}{x}  \Bigg\{ R_{\mu_1\mu_2}{}^\rho{}_{\mu_3} T^\sigma{}_{\mu_4\mu_5} \sum_{\pi \in  \mathrm{Z}_5}C_\pi\delta_\rho^{\mu_{\pi(1)}} \delta_\sigma^{\mu_{\pi(2)}} \epsilon^{\mu_{\pi(3)}\mu_{\pi(4)}\mu_{\pi(5)}}  + T^\rho{}_{\mu_1\mu_2} T^\sigma{}_{\mu_3\mu_4} T^\tau{}_{\mu_5\mu_6} \sum_{\pi \in \mathrm{Z}_6}D_\pi\delta_\rho^{\mu_{\pi(1)}} \delta_\sigma^{\mu_{\pi(2)}}\delta_\tau^{\mu_{\pi(3)}}\epsilon^{\mu_{\pi(4)}\mu_{\pi(5)}\mu_{\pi(6)}} + T^\rho{}_{\mu_1\mu_2}\nabla_{\mu_3} T^\sigma{}_{\mu_4\mu_5}\sum_{\pi \in \mathrm Z_5}E_\pi\delta_\rho^{\mu_{\pi(1)}} \delta_\sigma^{\mu_{\pi(2)}}\epsilon^{\mu_{\pi(3)}\mu_{\pi(4)}\mu_{\pi(5)}} \Bigg\}, 
    %\end{split}
  \end{dmath}
  where all possible permutations of $n$ elements $\pi \in \mathrm{Z}_n$ have been included in the sums with  different constants $C_\pi$, $D_\pi$ and $E_\pi$ for  permutation. 
\end{widetext}

The torsion field can be decomposed into invariant tensors respecting the symmetry,
\begin{equation}
  T^\sigma{}_{\mu\nu} = \epsilon_{\mu\nu\rho} T^{\sigma\rho} + A_{[\mu}\delta^\sigma{}_{\nu]},
\end{equation}
with a symmetric $T^{\sigma\rho}$ of density weight  $w = 1$, and \mbox{$A_\mu = T^\nu{}_{\mu\nu}$} is the trace part of the more arbitrary $T^\sigma{}_{\mu\nu}$.

The action in Eq.~\eqref{accion3d} can be rewritten as
\begin{dmath}
  \label{accion3dnew}
  S[\Gamma] =
  \int\dn{3}{x} \Bigg( 
  B_1R_{\mu\nu}{}^\mu{}_{\rho} T^{\nu\rho} 
  +B_2\epsilon^{\mu\nu\rho}R_{\mu\nu}{}^{\sigma}{}_\sigma A_\rho 
  +B_3 \epsilon^{\mu\nu\rho}R_{\mu\nu}{}^\lambda{}_\rho A_\lambda 
  +B_4 \det(T^{\mu\nu}) 
  +B_5 T^{\mu\nu}A_\mu A_\nu 
  +B_6 T^{\mu\nu}\nabla_\mu A_\nu
  +B_7\epsilon^{\mu\nu\rho}A_\mu\partial_\nu A_\rho
  + B_8\epsilon^{\mu\nu\rho}\Gamma^{\sigma}{}_{\mu\sigma}\partial_\nu\Gamma^{\tau}{}_{\rho\tau}
  +B_9\epsilon^{\mu\nu\lambda}\Big(\Gamma^{\sigma}{}_{\mu\rho}\partial_\nu\Gamma^{\rho}{}_{\lambda\sigma}
  +\frac{2}{3}\Gamma^{\tau}{}_{\mu\rho}\Gamma^{\rho}{}_{\nu\sigma}\Gamma^{\sigma}{}_{\lambda\tau}{}\Big)
  \Bigg).
\end{dmath}
where the coefficients $B_i$ are related to the original coefficients $C_i$, $D_i$ and $E_i$, and the additional $B_9$ term can be added in three dimensions leaving the action invariant under diffeomorphisms. The affine connection can be decomposed into its symmetric and antisymmetric parts, 
\begin{equation}
  \Gamma^\lambda{}_{\mu\rho}=\hat{\Gamma}^\lambda{}_{(\mu\rho)} + \epsilon_{\mu\rho\sigma}T^{\lambda\sigma} + A_{[\mu}\delta^\lambda{}_{\rho]},
\end{equation}
where  $\epsilon_{\mu\rho\sigma}$ has been introduced, and it is related to the skew symmetric $\epsilon^{\mu\rho\sigma}$ through the identity \mbox{$\epsilon^{\lambda\mu\nu}\epsilon_{\rho\sigma\tau}=3!\delta^{\lambda}{}_{[\rho}\delta^\mu{}_{\sigma}\delta^{\nu}{}_{\tau]}$.} Therefore, the curvature tensor can be expressed as 
\begin{dmath}
  \label{RiemmanDecomposition}
  R_{\mu\nu}{}^\sigma{}_\rho=
  \hat{R}_{\mu\nu}{}^\sigma{}_\rho
  -2\epsilon_{\rho\alpha[\mu}\hat\nabla_{\nu]}T^{\sigma\alpha}
  +\partial_{[\mu}A_{\nu]}\delta^\sigma_\rho
  +\delta^\sigma_{[\mu}\hat\nabla_{\nu]}A_\rho
  +\epsilon_{\mu\nu\kappa}T^{\kappa\sigma}A_\rho
  -\delta^\sigma_{[\mu}\epsilon_{\nu]\rho\alpha}T^{\alpha\beta}A_\beta 
  +\frac{1}{2}\delta^\sigma_{[\mu}A_{\nu]}A_\rho
  -2\epsilon_{\alpha\beta[\mu}\epsilon_{\nu]\rho\delta}T^{\sigma\alpha}T^{\beta\delta},
\end{dmath}
where $ \hat\nabla_\rho$ and $\hat{R}_{\mu\nu}{}^\lambda{}_\rho$ are the covariant derivative and  curvature associated to the symmetric part of the connection. Notice that Bianchi identity, obtained as $\epsilon^{\mu\nu\lambda}\hat R_{\mu\nu}{}^\rho{}_\lambda=0$, leads us to the following
\begin{equation}
  \label{bianchi}
  \epsilon^{\mu\nu\rho} R_{\mu\nu}{}^\lambda{}_\rho = 4\hat\nabla_\rho T^{\rho\lambda}
  +2\epsilon^{\mu\nu\lambda}\partial_\mu A_\nu-4T^{\lambda\rho}A_\rho. 
\end{equation}
Using the  Eqs.~\eqref{RiemmanDecomposition} and~\eqref{bianchi} one can rewrite the action in Eq.~\eqref{accion3dnew} as
\begin{dmath}
  \label{accion3dfinal} 
  S[\Gamma] = \int\dn{3}{x} \Bigg(  
  (B_1+2B_9) \hat R_{\mu\nu}{}^\mu{}_{\rho} T^{\nu\rho}   
  + (B_2+B_9+B_9) \epsilon^{\mu\nu\rho}\hat R_{\mu\nu}{}^{\sigma}{}_\sigma A_\rho  
  + (-6B_1+B_4-4B_9) \det(T^{\mu\nu})   
  + (\frac{1}{2}B_1-4B_3+B_5+B_9 ) T^{\mu\nu}A_\mu A_\nu   
  + (B_1-4B_3+B_6+2B_9) T^{\mu\nu}\hat\nabla_\mu A_\nu  
  + (2B_2+2B_3+B_7+B_9)\epsilon^{\mu\nu\rho}A_\mu\partial_\nu A_\rho  
  + B_9\epsilon^{\mu\nu\lambda}\Big(\hat\Gamma^{\sigma}{}_{\mu\rho}\partial_\nu\hat\Gamma^{\rho}{}_{\lambda\sigma}  + \frac{2}{3}\hat\Gamma^{\tau}{}_{\mu\rho}\hat\Gamma^{\rho}{}_{\nu\sigma}{}\hat\Gamma^{\sigma}{}_{\lambda\tau}{}\Big)  
  + B_8\epsilon^{\mu\nu\rho}\hat\Gamma^{\sigma}{}_{\mu\sigma}\partial_\nu\hat\Gamma^{\tau}{}_{\rho\tau}  
  + \partial_\alpha\Big( 4B_3T^{\alpha\mu} A_\mu +B_9(2\Gamma^{[\delta}{}_{\delta\beta}T^{\alpha]\beta}-T^{\alpha\beta}A_\beta+\frac{1}{2}\epsilon^{\beta\alpha\eta}A_\eta\Gamma^{\delta}{}_{\delta\beta})  +B_8\epsilon^{\beta\alpha\eta}A_\eta\hat\Gamma^{\delta}{}_{\delta\beta}\Big)  \Bigg),
\end{dmath}
or after dropping the boundary terms and rename the coefficients,
\begin{dmath}
  S[\hat\Gamma,T,A] =
  \int \dn{3}{x} \bigg( 
  A_1\hat{R}_{\mu\nu}{}^{\mu}{}_\rho T^{\nu\rho} 
  +A_2\epsilon^{\mu\nu\rho}\hat{R}_{\mu\nu}{}^{\sigma}{}_\sigma A_\rho
  +A_3\epsilon^{\mu\nu\rho}A_\mu\partial_\nu A_\rho
  +A_4T^{\mu\nu}\hat{\nabla}_\mu A_\nu
  +A_5T^{\mu\nu}A_\mu A_\nu
  +A_6\det(T^{\mu\nu}) 
  +A_7\epsilon^{\mu\nu\lambda}\Big(\hat{\Gamma}^{\sigma}{}_{\mu\rho}\partial_\nu\hat{\Gamma}^{\rho}{}_{\lambda\sigma}
  +\frac{2}{3}\hat{\Gamma}^{\tau}{}_{\mu\rho}\hat{\Gamma}^{\rho}{}_{\nu\sigma}{}\hat{\Gamma}^{\sigma}{}_{\lambda\tau}{}\Big)
  + A_8\epsilon^{\mu\nu\rho}\hat{\Gamma}^{\sigma}{}_{\mu\sigma}\partial_\nu\hat{\Gamma}^{\tau}{}_{\rho\tau}
  \bigg).
\end{dmath}

%Poplawski...
Noticing that in  the first term, the variation respect to the Ricci tensor yields to $T^{\mu\nu}$, it can be argued that in a standard theory of gravity this tensor density corresponds to $\sqrt{g}g^{\mu\nu}$.
Therefore, Eq.~\eqref{accion3dfinal} reveals a one to one correspondence with General Relativity non minimally coupled to the $A_\mu$ field,
\begin{dmath}
  \label{accion3dGR}
  S[g,\hat{\Gamma},A] = \int \dn{3}{x} \bigg(
  \sqrt{g} \Big(  A_1 \hat{R} + A_4\hat{\nabla}^\mu A_\mu + A_5 A_\mu A^\mu + A_6  \Big)
  + A_2\epsilon^{\mu\nu\rho} \hat{R}_{\mu\nu}{}^{\sigma}{}_\sigma A_\rho
  + A_3\epsilon^{\mu\nu\rho}A_\mu\partial_\nu A_\rho
  + A_7\epsilon^{\mu\nu\lambda}\Big(\hat{\Gamma}^{\sigma}{}_{\mu\rho}\partial_\nu\hat{\Gamma}^{\rho}{}_{\lambda\sigma}
  + \frac{2}{3}\hat{\Gamma}^{\tau}{}_{\mu\rho}\hat{\Gamma}^{\rho}{}_{\nu\sigma}{}\hat{\Gamma}^{\sigma}{}_{\lambda\tau}{}\Big)
  + A_8\epsilon^{\mu\nu\rho}\hat{\Gamma}^{\sigma}{}_{\mu\sigma}\partial_\nu\hat{\Gamma}^{\tau}{}_{\rho\tau}
  \bigg)
\end{dmath}
Thus, an interesting sector of the theory corresponds to the space of non-degenerated $T^{\mu\nu}$. 




\section{\label{sec:4} Four-dimensional metricless (and torsionful) action}

Following the precepts  already stated, we start  by defining an irreducible representation decomposition for the full connection field 
\begin{equation}
  \Gamma^\mu{}_{\rho\sigma} = \hat{\Gamma}^\mu{}_{\rho\sigma} + T^\mu{}_{\rho\sigma} = \hat{\Gamma}^\mu{}_{\rho\sigma} + \epsilon_{\rho\sigma\lambda\kappa}T^{\mu,\lambda\kappa}+A_{[\rho}\delta^\mu_{\nu]},
\end{equation}
where $\hat{\Gamma}^\mu{}_{\rho\sigma}$ denotes a $\tfrac{d^2(d+1)}{2}$ dimensional symmetric connection, $A_\mu$ is a $d$ dimensional vector field  that gives trace to the antisymmetric part of the full connection, and  $T^{\mu,\lambda\kappa}$ is a $\tfrac{d(d+1)(d-2)}{2}$ dimensional Curtright field (see Ref.~\cite{Curtright:1980yk}) that is defined through the symmetry of its indices: antisymmetric in the last two indices, and it has a cyclic property $T^{\mu,\lambda\kappa}+T^{\lambda,\kappa\mu}+T^{\kappa,\mu\lambda}=0$. In other words that $T^{[\mu,\lambda]\kappa}=\frac{1}{2}T^{\kappa,\lambda\mu}$, just as for  the Riemmann tensor $\hat{R}_{\mu[\nu}{}^\alpha{}_{\lambda]}=\frac{1}{2}\hat{R}_{\lambda\nu}{}^\alpha{}_{\mu}$. Notice that due to its symmetries, the contraction $\epsilon_{\rho\sigma\lambda\kappa}T^{\mu,\lambda\kappa}$ is traceless.

Additionally, since no metric is present  the epsilon symbols are not related by lowering or raising their indices, but instead one demands that \mbox{$\epsilon^{\delta\eta\lambda\kappa}\epsilon_{\mu\nu\rho\sigma}=4!\delta^{\delta}{}_{[\mu}\delta^\eta{}_{\nu}\delta^{\lambda}{}_{\rho} \delta^\kappa{}_{\sigma]}$.}

\begin{widetext}
  One can  write all the combinations of fields that would presumably be renormalizable with these three independent fields
  \begin{dmath}[compact, spread=2pt] 
    \label{4dfull}
    S[\hat{\Gamma},T,A] =
    \int\dn{4}{x}\Bigg[\partial_\lambda\bigg(
    A_1\hat{ R}_{\mu(\nu}{}^{\mu}{}_{\rho)} T^{\nu,\rho\lambda} 
    +A_2\epsilon^{\lambda\mu\nu\rho}\hat{ R}_{\mu\nu}{}^{\sigma}{}_\sigma A_\rho
    +A_3\epsilon^{\lambda\mu\nu\rho}A_\mu\partial_\nu A_\rho
    +A_4T^{(\mu,\nu)\lambda}\hat\nabla_\mu A_\nu
    +A_5T^{(\mu,\nu)\lambda}A_\mu A_\nu
    +A_6\epsilon_{\mu\nu\rho\sigma}\epsilon_{\alpha\beta\gamma\delta} T^{\lambda,\mu\alpha} T^{\beta,\rho\sigma} T^{\nu,\gamma\delta} 
    +A_7\epsilon^{\lambda\mu\nu\lambda}\Big(\hat\Gamma^{\sigma}{}_{\mu\rho}\partial_\nu\hat\Gamma^{\rho}{}_{\lambda\sigma}
    +\frac{2}{3}\hat\Gamma^{\tau}{}_{\mu\rho}\hat\Gamma^{\rho}{}_{\nu\sigma}{}\hat\Gamma^{\sigma}{}_{\lambda\tau}{}\Big) 
    + A_8\epsilon^{\lambda\mu\nu\rho}\hat\Gamma^{\sigma}{}_{\mu\sigma}\partial_\nu\hat\Gamma^{\tau}{}_{\rho\tau}
    +A_9 R_{\mu\nu}{}^{\lambda}{}_{\rho} T^{\rho,\mu\nu}
    +A_{10}T^{\lambda,\alpha\beta}T^{\kappa,\gamma\delta} A_\kappa\epsilon_{\alpha\beta\gamma\delta}
    \bigg)
    +B_1 \hat R_{\mu\nu}{}^{\mu}{}_{\rho} T^{\nu,\alpha\beta}T^{\rho,\gamma\delta}\epsilon_{\alpha\beta\gamma\delta}
    +B_2 \Big(\hat R_{\mu\nu}{}^{\sigma}{}_\rho+\frac{2}{3}\delta^\sigma{}_{[\mu}\hat R_{\nu]\lambda}{}^{\lambda}{}_\rho \Big) T^{\beta,\mu\nu}T^{\rho,\gamma\delta}\epsilon_{\sigma\beta\gamma\delta}
    +B_3 \hat R_{\mu\nu}{}^{\mu}{}_{\rho} T^{(\nu,\rho)\sigma}A_\sigma
    + B_4\Big(\hat R_{\mu\nu}{}^{\sigma}{}_\rho+\frac{2}{3}\delta^\sigma{}_{[\mu}\hat R_{\nu]\lambda}{}^{\lambda}{}_\rho \Big)\Big(T^{\rho,\mu\nu}A_\sigma-\frac{1}{4}\delta^\rho_\sigma T^{\kappa,\mu\nu}A_\kappa\Big)
    +B_5\hat R_{\mu\nu}{}^{\rho}{}_\rho T^{\sigma,\mu\nu}A_\sigma
    +C_1 \hat R_{\mu\nu}{}^{\mu}{}_{\rho} \hat\nabla_\sigma T^{(\nu,\rho)\sigma}
    +C_2\Big(\hat R_{\mu\nu}{}^{\sigma}{}_\rho+\frac{2}{3}\delta^\sigma{}_{[\mu}\hat R_{\nu]\lambda}{}^{\lambda}{}_\rho \Big)\Big(\hat\nabla_\sigma T^{\rho,\mu\nu}-\frac{1}{4}\delta^\rho_\sigma \hat\nabla_\kappa T^{\kappa,\mu\nu}\Big)
    +C_3\hat R_{\mu\nu}{}^{\rho}{}_\rho \hat\nabla_\sigma T^{\sigma,\mu\nu} 
    +D_1T^{\alpha,\mu\nu}T^{\beta,\rho\sigma}\hat\nabla_\gamma T^{(\lambda, \kappa) \gamma}\epsilon_{\beta\mu\nu\lambda}\epsilon_{\alpha\rho\sigma\kappa}
    +D_2T^{\alpha,\mu\nu}T^{\lambda,\beta\gamma}\hat\nabla_\lambda T^{\delta,\rho\sigma}\epsilon_{\alpha\beta\gamma\delta}\epsilon_{\mu\nu\rho\sigma}
    +D_3T^{\mu,\alpha\beta}T^{\lambda,\nu\gamma}\hat\nabla_\lambda T^{\delta,\rho\sigma}\epsilon_{\alpha\beta\gamma\delta}\epsilon_{\mu\nu\rho\sigma}
    +D_4T^{\lambda,\mu\nu}T^{\kappa,\rho\sigma}\hat\nabla_{(\lambda} A_{\kappa)} \epsilon_{\mu\nu\rho\sigma}
    +D_5T^{\lambda,\mu\nu}\hat\nabla_{[\lambda}T^{\kappa,\rho\sigma} A_{\kappa]} \epsilon_{\mu\nu\rho\sigma}
    +D_6T^{\lambda,\mu\nu}A_\nu\hat\nabla_{(\lambda} A_{\mu)}
    +D_7T^{\lambda,\mu\nu}A_\lambda\hat\nabla_{[\mu} A_{\nu]} 
    +E_1\hat\nabla_{(\rho} T^{\rho,\mu\nu}\hat\nabla_{\sigma)} T^{\sigma,\lambda\kappa}\epsilon_{\mu\nu\lambda\kappa}
    +E_2\hat\nabla_{(\lambda} T^{\lambda,\mu\nu}\hat\nabla_{\mu)} A_\nu
    +T^{\alpha,\beta\gamma}T^{\delta,\eta\kappa}T^{\lambda,\mu\nu}T^{\rho,\sigma\tau}
    \big(\Lambda_1\epsilon_{\beta\gamma\eta\kappa}\epsilon_{\alpha\rho\mu\nu}\epsilon_{\delta\lambda\sigma\tau}
    +\Lambda_2\epsilon_{\beta\lambda\eta\kappa}\epsilon_{\gamma\rho\mu\nu}\epsilon_{\alpha\delta\sigma\tau}\big) 
    +\Lambda_3 T^{\rho,\alpha\beta}T^{\gamma,\mu\nu}T^{\lambda,\sigma\tau}A_\tau \epsilon_{\alpha\beta\gamma\lambda}\epsilon_{\mu\nu\rho\sigma}
    +\Lambda_4T^{\eta,\alpha\beta}T^{\kappa,\gamma\delta}A_\eta A_\kappa\epsilon_{\alpha\beta\gamma\delta}\Bigg],
  \end{dmath}
  where the $A_i$ series of terms contribute solely to the boundary conditions, and the terms $B_2$, $B_4$ and $C_2$ contain a traceless contribution of the curvature.
\end{widetext}
In this case, the induced  ``inverse metric density'' [see Eq.~\eqref{metric}] is 
\begin{dmath}
  \label{4dMetric}
  \bar{g}^{\mu\nu} \equiv \sqrt{-g}g^{\mu\nu} = B_1 T^{\mu,\lambda\kappa}T^{\nu,\rho\sigma}\epsilon_{\lambda\kappa\rho\sigma} + B_3 T^{(\mu,\nu)\lambda}A_\lambda + C_1 \hat{\nabla}_\lambda T^{(\mu,\nu)\lambda}.
\end{dmath}


\subsection*{Symmetric solution to the equations of motion}

In four dimensions there is no obvious equivalence of Eq.~\eqref{4dfull} with GR, specially due to the lack of a fundamental metric field in the given model. However, both models are explicitly invariant under diffeomorphisms, and even if their structures and number of degrees of freedom differ, the action in Eq.~\eqref{4dfull} provides a context where parallel transport of particle's velocities on a purely torsional background is nontrivial.

Here, we wish to stablish the model's non-relativistic (Newtonian) limit for the ``geodesic" deviation of ``inertial" observers at rest with respect to a static, isotropic, homogeneous and spatially flat background within the context provided by Eq.~\eqref{4dfull}. In order to properly analyse the model, we  propose the following decomposition of the fields
\begin{align}
  A_\mu &= \delta_\mu^0 A + a_\mu,\\
  T^{\mu,\nu\rho} &= \delta^{\mu}_m\delta^{\nu\rho}_{m0}T + t^{\mu,\nu\rho},\\
  \shortintertext{and}
  \hat{\Gamma}^\lambda{}_{\mu\nu} &= E \delta^\lambda_0 \delta^m_\mu \delta^m_\nu + F \delta^\lambda_m \delta^m_{(\mu}\delta^0_{\nu)} + G\delta^\lambda_0 \delta^0_{\mu}\delta^0_{\nu} + \gamma^\lambda{}_{\mu\nu},
\end{align}
where $\delta^{\mu\nu}_{\lambda\kappa}=\delta^{\mu}_{\lambda}\delta^{\nu}_{\kappa}-\delta^{\mu}_{\kappa}\delta^{\nu}_{\lambda}$.


In order to make perturbation theory we will expand around a static, isotropic and homogeneous solution of the equations of motion, because these are characteristic of the observable universe.

The induced metric in Eq.~\eqref{4dMetric}  on the background is
\begin{dmath}
  \label{3+1metric}
  \sqrt{-g}g^{\mu\nu} = \left(B_3 A + \frac{1}{2}C_1 F\right) T \delta^\mu_m \delta^\nu_m - 3 C_1 E T \delta^\mu_0\delta^\nu_0,
\end{dmath}
while the Ricci curvature tensor is
\begin{dmath}
  R_{\mu\nu} = \frac{1}{2} E F \delta^m_\mu \delta^m_\nu - \frac{3}{4} F^2 \delta^0_\mu \delta^0_\nu.
\end{dmath}
Therefore, whether the four-dimensional metric structure is Riemannian or pseudo-Riemannian will depend exclusively on the values of the parameters of the action in Eq.~\eqref{4dfull} and the signs of the components of the connection field. 
\begin{widetext}
  The first order perturbations of the action yields
  \begin{dmath}[compact, spread=2pt]
    \label{EOM0thOrder}
    \delta S =
    \bigg( \Big( ( B_3 + \frac{8}{3}\, B_4 + \frac{1}{2}\, E_2) A + (4\, C_1 - \frac{16}{3}\, C_2) F + ( - 2\, C_1 + \frac{8}{3}\, C_2) G \Big) E + 8\, ( - D_1 + 2\, D_2 + D_3) T T \bigg) T \delta{\Gamma}^{m}\,_{0 m}
    + \bigg( ( \frac{1}{2}\, B_3 + \frac{4}{3}\, B_4 + \frac{1}{4}\, E_2) A F + ( B_3 - \frac{4}{3}\, B_4 - \frac{1}{2}\, E_2) A G + (C_1 - \frac{4}{3}\, C_2) F F + ( - C_1 + \frac{4}{3}\, C_2) F G - D_6 A A \bigg) T \delta{\Gamma}^{0 m}\,_{m}
    + \bigg( \Big(- (\frac{1}{2}\, B_3 + \frac{4}{3}\, B_4 + \frac{1}{4}\, E_2) A F + ( - B_3+ \frac{4}{3}\, B_4 + \frac{1}{2}\, E_2) A G + ( - C_1 + \frac{4}{3}\, C_2) F F + (C_1 - \frac{4}{3}\, C_2) F G + D_6 A A \Big) E + \Big( 12\, ( D_1 - 2\, D_2 - D_3) F + 24\, L_3 A \Big) T T \bigg)\delta{T}_{m}\,^{0 m}
    + \bigg( ( 3\, B_3 - 4\, B_4 - \frac{3}{2}\, E_2) A + ( - 3\, C_1 + 4\, C_2) F \bigg) E T \delta{\Gamma}^{0}\,_{0 0}
    + \bigg( 3\Big( - 2\, D_6 A + ( \frac{1}{2}\, B_3 + \frac{4}{3}\, B_4 + \frac{1}{4}\, E_2) F + ( B_3 - \frac{4}{3}\, B_4 - \frac{1}{2}\, E_2) G \Big) E - 24\, L_3 T T \bigg) T \delta{A}_{0}=0,
  \end{dmath}
\end{widetext}
and we are most interested in solutions to the connection field whose contribution to the parallel transport equation of a test particle's velocity is that of a free particle, at least at the low velocity regime
\begin{equation} 
  \ddot{x}^i+2F\dot{x}^0\dot{x}^i=0, \quad \text{and} \quad \ddot{x}^0 + E \, (\dot{x}^i)^2 + G \, (\dot{x}^0)^2 = 0,
\end{equation}
which we can achieve by setting $F=G=0$ and $E \neq 0$ since $(\dot{x}^i)^2$ is already second order in the velocities. Thus, looking again at the equations of motion we can find a non trivial solution if we set all coupling constants to zero but $B_3 \neq 0$, $B_4 = -\tfrac{3}{2} B_3$, $C_1\neq 0$ and $E_2= 6 B_3$. 

Additionally, we can incorporate perturbative inhomogeneous sources to the connection field equations and check on how these fluctuations affect motion. For this, we consider a matter's action, whose dependence on the affine connection can be almost arbitrary. However, we will presume that it will depend only on the barred metric in Eq.~\eqref{4dMetric}
$$ S_{\text{Matter}} = {S}_{\text{Matter}}[\bar{g}^{\mu\nu}].$$
Thus, a non-moving matter point particle at the origin of the reference frame will contribute to the equations of motion for the gravitational field through the component $\bar{g}^{00}$ following the symmetries of the matter source 
\begin{dmath}
  \label{mattervariation}
  \delta {S}_{\text{Matter}} =  C_1 \Big(- \frac{1}{2} ({\delta\Gamma}^{0}{}_{m n})  T {\delta}^{m n} - \frac{1}{2}  ({\delta T}^{0 0 m})  \imath {p}_{m} + \frac{1}{2}  ({\delta T}^{m 0 n})  E {\delta}_{m n} \Big)\frac{\partial\mathcal{L}_{\text{Matter}}}{\partial \bar{g}^{00}}.
\end{dmath}


\subsection*{Scalar modes  and Newtonian limit}

In order to obtain the non-relativistic limit, \emph{i.e.}, the Newtonian potential, one performs the scalar mode perturbative expansion. One proceeds by substituting the connection and torsion components by their scalar perturbation decomposition,
\begin{equation}
  a_\mu \to \delta_\mu^0 a+\delta_\mu^m \partial_{m}a_L,
\end{equation}
\mbox{}
\begin{dmath}
  t^{\mu,\nu\rho} \to \delta^{\mu}_m\delta^{\nu\rho}_{n0} \Big(t \delta^{m n} + \partial^m \partial^n t_L \Big)
  +\delta^{\mu}_0 \delta^{\nu\rho}_{m0} \partial^m c_L
  + \Big(\delta^{\mu}_0\delta^{\nu\rho}_{mn}-\delta^{\mu}_m\delta^{\nu\rho}_{n0}\Big)\epsilon^{m n p} \partial_{p} b
  +\delta^{\mu}_m \delta^{\nu}_{n} \delta^{\rho}_{p} \Big(\epsilon^{n p q}\partial_q \partial^m d_1 +  (\delta^{m n} \partial^p - \delta^{m p} \partial^n)d_2\Big)
\end{dmath}
and

\begin{dmath}
  \gamma^\lambda_{\mu\nu} \to
  \delta^\lambda_0\delta^0_\mu\delta^0_\nu u 
  + \delta^\lambda_m \delta^0_\mu\delta^0_\nu \partial^m v_L
  + 2\delta^\lambda_0 \delta^0_{(\mu}\delta^m_{\nu)} \partial_m w_L
  + \delta^\lambda_0 \delta^m_\mu\delta^n_\nu \Big(x \delta_{mn} + \partial_m \partial_n x_L\Big)
  + 2\delta^\lambda_m \delta^0_{(\mu}\delta^n_{\nu)} \Big(y_1 \delta^m{}_n + \epsilon^{m p}{}_{n} \partial_p y_2 + \partial^m \partial_n y_L\Big)
  + \delta^\lambda_m \delta^n_{\mu}\delta^p_{\nu} \Big(\delta_{n p} \partial^m z_1 + (\delta^m{}_n \partial_p+\delta^m{}_p \partial_n) z_2 +  (\epsilon^{m q}{}_n \partial_p+\epsilon^{m q}{}_p \partial_n) \partial_q z_3 + \partial^m \partial_n \partial_p z_L\Big),
\end{dmath}
where the scalar fields identified with the sub-index ``L'' correspond to longitudinal degrees of freedom. Vector and tensor perturbations are left for further investigations of the structure of the model.

The first order perturbative expansion of the equations of motion around the already described background in momentum space with $p_0=0$ is given by
\begin{widetext}
  \begin{dmath}
    \delta S =
    \Big( - 2 E d_2 + t - p^2 t_L + T w_L + 3 T z_1 + 2 T z_2 - T p^2 z_L \Big) 6 B_3 p^2 {\,\delta a}_{0} 
    + \bigg( - E p^2 d_2 + \frac{1}{2} p^2 t - \frac{1}{2} p^4 t_L - \frac{1}{2} T p^2 w_L - 6 E T y_1 + 2 E T p^2 y_L + \frac{3}{2} T p^2 z_1 + T p^2 z_2 - \frac{1}{2} T p^2 p^2 z_L \bigg) C_1 {\,\delta\gamma}^{0}{}_{0 0} 
    + \bigg(6 B_3 a - \frac{1}{2} C_1 u + 2 C_1 E v_L + \frac{1}{2} C_1 y_1 - \frac{3}{2} C_1 p^2 y_L \bigg) T \imath {p}^{m} {\,\delta\gamma}^{0}{}_{0 m} 
    + C_1 T p^2 v_L {\,\delta}^{m n} {\,\delta\gamma}^{0}{}_{m n} 
    + \Big( - p^2 c_L + 2 E E d_2 + 4 E t - 2 E p^2 t_L + 2 E T w_L + 3 T x - T p^2 x_L - 10 E T z_2 + 2 E T p^2 z_L \Big) C_1 \imath {p}_{m} {\,\delta\gamma}^{m}{}_{0 0} 
    + \bigg( E p^2 d_2 +  \frac{1}{2} p^2 t - \frac{1}{2} p^4 t_L - 2 E T u - \frac{1}{2} T p^2 w_L + 8 E T y_1 - 2 E T p^2 y_L + \frac{1}{2} T p^2 z_1 + T p^2 z_2 - \frac{1}{2} T p^4  z_L \bigg) C_1 {\,\delta}^{m}{}_{n} {\,\delta\gamma}^{n}{}_{0 m} 
    + \Big( - 4 E d_2  - t +  p^2 t_L + 2 T w_L - 2 E T y_L - 2 T z_1 + T p^2 z_L \Big)  C_1 {p}_{n} {p}^{m} {\,\delta\gamma}^{n}{}_{0 m} 
    - 6  E T y_2 C_1 \imath {p}^{p} {\epsilon}_{n p}{}^{m} {\,\delta\gamma}^{n}{}_{0 m} 
    + \bigg(6 B_3 T a + C_1 p^2 d_2 + \frac{1}{2} C_1 T u + C_1 E T v_L - \frac{1}{2} C_1 T y_1 + \frac{1}{2} C_1 T p^2 y_L \bigg) \imath {p}_{m} {\,\delta}^{n p} {\,\delta\gamma}^{m}{}_{n p} 
    + \Big( - 3 E v_L + y_1 - p^2 y_L\Big) C_1 T \imath {p}^{n} {\,\delta}_{m}{}^{p} {\,\delta\gamma}^{m}{}_{n p} 
    + \Big( - d_2 + T y_L \Big) C_1 \imath {p}_{m} {p}^{n} {p}^{p} {\,\delta\gamma}^{m}{}_{n p} 
    + v_L C_1 p^2 \imath {p}_{m} {\,\delta t}^{0 0 m} 
    - v_L C_1 E p^2 {\,\delta}_{m n} {\,\delta t}^{m 0 n} 
    + \bigg( - 6  B_3 a - \frac{1}{2}  C_1 u - C_1 E v_L - \frac{1}{2}  C_1 y_1 - \frac{1}{2}  C_1 p^2 y_L \bigg) {p}_{m} {p}_{n} {\,\delta t}^{m 0 n} 
    + \bigg(6 B_3 E a + \frac{1}{2} C_1 E u - C_1 E E v_L + \frac{1}{2} C_1 E y_1 - \frac{3}{2} C_1 E p^2 y_L - C_1 p^2 z_1 \bigg) \imath {p}_{m} {\,\delta}_{n p} {\,\delta t}^{n m p}.
  \end{dmath}
\end{widetext}
which we will add to the variations of the action of the matter from Eq.~\eqref{mattervariation}
%% \begin{dmath}
%%   \label{mattervariation}
%% \delta{\mathcal L}_{\text{Matter}} =  \Big(- \frac{1}{2} {dg}^{0} _{m n} C1 T {\delta}^{m n} - \frac{1}{2}  {dt}^{0 0 m} C1 i {p}_{m} + \frac{1}{2}  {dt}^{m 0 n} C1 E {\delta}_{m n}\Big)\frac{\partial{\mathcal L}_{\text{Matter}}}{\partial \bar{g}^{00}}
%% \end{dmath}
and set $\delta S_{\text{total}}=0$.

Solutions to this set of equations are in general a highly difficult problem that concerns $20$ equations of motion and $20$ scalar fields to be fixed by the equations. Yet, knowledge of the value of some of these scalars does not necessarily help to determine how geodesics are affected. For this, we only need $\gamma^i{}_{00}$ and $\gamma^0{}_{00}$ as these provide the first order contributions to the equations
\begin{equation}
  \label{geodesic1stOrder}
  \ddot{x}^i + \gamma^i{}_{00}(\dot{x}^0)^2 = 0,
  \quad \text{and} \quad
  \ddot{x}^0 + \gamma^0{}_{00}(\dot{x}^0)^2 = 0.
\end{equation}
From Eq.~\eqref{geodesic1stOrder} we can conclude that we need only know $\gamma^i{}_{00} = \partial^i v_L$, which we obtain in Fourier space to be

\begin{equation}
  v_L=\frac{1}{2}\frac{\partial\mathcal{L}_{\text{Matter}}}{\partial\bar{g}^{00}}\frac{1}{p^2}.
\end{equation}
In position space, 
\begin{equation}
  v_L = \frac{1}{8\pi} \frac{ \partial\mathcal{L}_{\text{Matter}} }{ \partial \bar{g}^{00} } \frac{1}{|\vec{x}|}
\end{equation}
is the usual Newtonian potential for a massive far off source.


%% \section{Matter fields}

%% Until now the most general diffeomorphism invariant and power counting renormalizable for an affine connection has been constructed. The theory does not possess a metric of the spacetime, instead one can find an induced inverse metric density arising as the functional variation of the action with respect to the symmetric part of the Ricci tensor.

%% Regarding to the equivalence of Einstein's theory of gravity to the action in Eq.~\eqref{4dfull},  no definite statement is made because \out{the number of local degrees of freedom disagree and} we have not been able to make out such correspondence. However, one of the major issues one could think of, is that even if there is an equivalence, the description of matter fields with half integer spin would be missing in a theory with no metric. The aim of this section is to describe the inclusion of Dirac spinor without a local Lorentz symmetry.

\section{\label{sec:dis} Discussion}

\begin{acknowledgements}
  We thank to J. Zanelli, 
\end{acknowledgements}


\nocite{%WorldSpinors,Tucker:1996sx,Horava:2009uw,Lu:2009em,
  Gibbs:1995gj,Peeters:2007wn,peeters2007symbolic,Peeters2007550,sage}

\bibliographystyle{apsrev4-1}
\bibliography{References.bib}

\end{document}

