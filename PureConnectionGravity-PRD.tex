\documentclass[twocolumn,
  showpacs,showkeys,prd,superscriptaddress]{revtex4-1}

\usepackage{amsmath,amsthm,latexsym,amssymb,amsfonts,epsfig}
\usepackage{xcolor}
\usepackage{ulem}
%\usepackage{authblk}
\usepackage[%
  colorlinks=true,
  urlcolor=blue,
  linkcolor=blue,
  citecolor=blue
]{hyperref}
\usepackage{etoolbox}
\usepackage{breqn}
\makeatletter
\let\cat@comma@active\@empty
\makeatother

%------------------
%--------- Definitions
%------------------
%% \def\be{\begin{equation}}
%% \def\ee{\end{equation}}
%% \def\ba{\begin{eqnarray}}
%% \def\ea{\end{eqnarray}}
\def\md{{\mathrm{d}}}
%\def\mx{{\mathrm{x}}}
\def\my{{\mathrm{y}}}
\def\mz{{\mathrm{z}}}
\def\l{\left}
\def\r{\right}
\def\ll{\left\{}
\def\rl{\right\}}
\def\lp{\left(}
\def\rp{\right)}
\def\lc{\left[}
\def\rc{\right]}
%% \renewcommand{\{}{\left\{}
%% \renewcommand{\}}{\right\}}
%% \renewcommand{\[}{\left[}
%% \renewcommand{\]}{\right]}
%% \renewcommand{\(}{\left(}
%% \renewcommand{\)}{\right)}
\def\la{\left\langle}
\def\ra{\right\rangle} 
%% \DeclareMathOperator{\det}{det}
%% \newcommand\tor[3]{\ensuremath{T_{#1}{}^{#2}{}_{#3}}}
%% %\renewcommand\Authands{ and }
%% \newcommand{\out}[1]{{\color{red} \sout{#1}}}
\newcommand{\hl}[1]{{\color{red} \textbf{#1}}}
%% \newcommand{\pro}[1]{{\color{blue} #1}}

\usepackage{amsmath,amssymb,amsfonts,dsfont,mathrsfs,amsthm}
\usepackage{graphicx}
\usepackage{centernot}
\usepackage{hyperref}
\usepackage{xcolor}
%% \usepackage{comment}
\hypersetup{linktocpage,colorlinks=true,urlcolor=blue!80!red,linkcolor=blue,citecolor=red}
%% \usepackage{feynmf}
\usepackage{siunitx}
\usepackage{array}
\usepackage{ulem}
%% \usepackage{tikz}
\usepackage{braket}
%% \usetikzlibrary{shapes,
%%   snakes,
%%   decorations.pathmorphing,
%%   decorations.markings,
%%   calc,
%%   shadows.blur,
%%   shadings}
%% \usepackage[framemethod=tikz]{mdframed}

%% \makeatletter
%% % gluon decoration (based on the original coil decoration)
%% \pgfdeclaredecoration{gluon}{coil}
%% {
%%   \state{coil}[switch if less than=%
%%     0.5\pgfdecorationsegmentlength+%>
%%     \pgfdecorationsegmentaspect\pgfdecorationsegmentamplitude+%
%%     \pgfdecorationsegmentaspect\pgfdecorationsegmentamplitude to last,
%%                width=+\pgfdecorationsegmentlength]
%%   {
%%     \pgfpathcurveto
%%     {\pgfpoint@oncoil{0    }{ 0.555}{1}}
%%     {\pgfpoint@oncoil{0.445}{ 1    }{2}}
%%     {\pgfpoint@oncoil{1    }{ 1    }{3}}
%%     \pgfpathcurveto
%%     {\pgfpoint@oncoil{1.555}{ 1    }{4}}
%%     {\pgfpoint@oncoil{2    }{ 0.555}{5}}
%%     {\pgfpoint@oncoil{2    }{ 0    }{6}}
%%     \pgfpathcurveto
%%     {\pgfpoint@oncoil{2    }{-0.555}{7}}
%%     {\pgfpoint@oncoil{1.555}{-1    }{8}}
%%     {\pgfpoint@oncoil{1    }{-1    }{9}}
%%     \pgfpathcurveto
%%     {\pgfpoint@oncoil{0.445}{-1    }{10}}
%%     {\pgfpoint@oncoil{0    }{-0.555}{11}}
%%     {\pgfpoint@oncoil{0    }{ 0    }{12}}
%%   }
%%   \state{last}[next state=final]
%%   {
%%     \pgfpathcurveto
%%     {\pgfpoint@oncoil{0    }{ 0.555}{1}}
%%     {\pgfpoint@oncoil{0.445}{ 1    }{2}}
%%     {\pgfpoint@oncoil{1    }{ 1    }{3}}
%%     \pgfpathcurveto
%%     {\pgfpoint@oncoil{1.555}{ 1    }{4}}
%%     {\pgfpoint@oncoil{2    }{ 0.555}{5}}
%%     {\pgfpoint@oncoil{2    }{ 0    }{6}}
%%   }
%%   \state{final}{}
%% }

%% \def\pgfpoint@oncoil#1#2#3{%
%%   \pgf@x=#1\pgfdecorationsegmentamplitude%
%%   \pgf@x=\pgfdecorationsegmentaspect\pgf@x%
%%   \pgf@y=#2\pgfdecorationsegmentamplitude%
%%   \pgf@xa=0.083333333333\pgfdecorationsegmentlength%
%%   \advance\pgf@x by#3\pgf@xa%
%% }
%% \makeatother

%% \tikzset{
%%   boson/.style={decorate,decoration={gluon,segment length=9pt,aspect=0}},
%%   % style to apply some styles to each segment of a path
%%   on each segment/.style={
%%     decorate,
%%     decoration={
%%       show path construction,
%%       moveto code={},
%%       lineto code={
%%         \path [#1]
%%         (\tikzinputsegmentfirst) -- (\tikzinputsegmentlast);
%%       },
%%       curveto code={
%%         \path [#1] (\tikzinputsegmentfirst)
%%         .. controls
%%         (\tikzinputsegmentsupporta) and (\tikzinputsegmentsupportb)
%%         ..
%%         (\tikzinputsegmentlast);
%%       },
%%       closepath code={
%%         \path [#1]
%%         (\tikzinputsegmentfirst) -- (\tikzinputsegmentlast);
%%       },
%%     },
%%   },
%%   % style to add an arrow in the middle of a path
%%   mid arrow/.style={postaction={decorate,decoration={
%%         markings,
%%         mark=at position .5 with {\arrow[#1]{stealth}}
%%       }}},
%% }


%-------------------------------Theorems
\newtheorem{Def}{Definition}
\newtheorem{Thm}{Theorem}
\newtheorem{Lem}{Lemma}
\newtheorem{Pos}{Postulate}
\newtheorem{Exa}{Example}
\newtheorem{Cor}{Corrolary}
\newtheorem{Pro}{Proposition}

%---------------------------------New commands
\newcommand{\A}{\mathcal{A}} 
\newcommand{\abs}[1]{\left|{#1}\right|}
\newcommand{\Ag}{\mathcal{A}_{(1)}}
\newcommand{\Agf}{\boldsymbol{\mathcal{A}}}
\newcommand{\Af}{\, {\mathbf{A}} }
\newcommand{\AF}[1]{\, {\mathbf{A}}_{({#1})} }
\newcommand{\hAf}{\, \hat{\mathbf{A}}_{(1)} }
\newcommand{\hAF}[1]{\, \hat{\mathbf{A}}_{(#1)} }
\newcommand{\bboxed}[1]{{\color{red}{\boxed{\boxed{\textcolor{black}{#1}}}}}}
\newcommand{\C}{\mathbb{C}}
\newcommand{\Cl}{\mathcal{C}\!\ell}
\newcommand{\cdf}[1][]{\,{\boldsymbol{\mathcal{D}}}{#1}\!}
\newcommand{\covd}{\mathcal{D}}
\newcommand{\D}{\mathscr{D}}
\newcommand{\df}[1][]{\,{\mathbf{d}}{#1}\!}
\newcommand{\dfd}{\,{\mathbf{d}}^\dag\!}
\newcommand{\ele}[2][]{\frac{d}{dt}\left(\frac{\partial\mathcal{L}}{\partial \dot{#2}^{#1}}\right) - \frac{\partial\mathcal{L}}{\partial {#2}^{#1}}}
\newcommand{\fele}[2][]{\partial_\mu \left(\frac{\delta\mathcal{L}}{\delta\left(\partial_\mu{#2}^{#1}\right)}\right) - \frac{\delta\mathcal{L}}{\delta {#2}^{#1}}}
\newcommand{\vb}[1]{\vec{e}_{#1}}
%% \newcommand{\fb}[1]{\widetilde{e}{}^{\, #1}}
\newcommand{\fb}[1]{\widetilde{e}{}^{\, #1}}
\newcommand\fder[3][]{\frac{\delta^{#1}{#2}}{\delta {#3}^{#1}}}
\newcommand\fdern[4][]{\frac{\delta^{#1}{#2}}{\delta {#3} \cdots \delta {#4}}}
\newcommand{\F}{\,\boldsymbol{\mathcal{F}}}
\newcommand{\Fg}{\,\boldsymbol{\mathcal{F}}_{(2)}}
\newcommand{\Ff}{\,{\mathbf{F}}}
\newcommand{\FF}[1]{\,{\mathbf{F}}_{(#1)}}
\newcommand{\hFF}[1]{\,{\mathbf{\hat{F}}}_{(#1)}}
\newcommand{\fy}{\centernot}
\newcommand{\G}{\mathscr{G}}
\newcommand{\ga}{\gamma}
\newcommand{\gf}{\,\boldsymbol{\gamma}}
\newcommand{\Ga}{\Gamma}
\newcommand{\conn}[3]{\left(\Gamma_{#1}\right)^{#2}{}_{#3}}
\newcommand{\Ha}{\mathscr{H}}
\newcommand{\He}{\mathbb{H}}
\newcommand{\Hi}{\mathcal{H}}
\newcommand{\Hint}{\underline{\sc Hint:} }
\DeclareMathOperator{\hs}{\,\star\!\!}
\newcommand{\J}{\mathscr{J}}
\newcommand{\K}{\mathbb{K}}
\newcommand{\KK}{Kaluza-Klein{\;}}
\newcommand{\Lag}{\mathscr{L}}
\newcommand{\Li}{\mathcal{L}}
\newcommand{\La}[1][]{\triangle_{#1}}
\newcommand{\Lap}{\nabla^2}
\newcommand{\lr}[1]{\stackrel{\leftrightarrow}{#1}}
\newcommand{\M}{\ensuremath{\mathscr{M}}}
\newcommand{\Mi}{\mathcal{M}}
\newcommand{\MN}{Maldacena-N\'u\~nez{\;}}
\newcommand{\N}{\ensuremath{\mathscr{N}}}
\newcommand{\Na}{\mathbb{N}}
\newcommand{\No}{\mathcal{N}}
\newcommand{\norm}[1]{\left\|#1\right\|}
\newcommand{\Op}{\mathcal{O}}
\newcommand{\Or}{\mathscr{O}}
\newcommand\pder[3][]{\frac{\partial^{#1}{#2}}{\partial {#3}^{#1}}}
\newcommand\pdern[4][]{\frac{\partial^{#1}#2}{\partial #3\cdots\partial #4}}
\newcommand{\Qh}[1][]{\ensuremath{\hat{Q}_{#1}}}
\newcommand{\R}{\mathbb{R}}
\newcommand{\Ri}{\mathcal{R}}
%% \newcommand{\S}{\mathscr{S}}
\newcommand{\SM}{Standard Model {}}%\mathscr{S}}
\newcommand{\T}{\mathscr{T}}
\newcommand{\tor}{\mathcal{T}}
\newcommand{\tors}[3]{\mathcal{T}{}_{#1}{}^{#2}{}_{#3}}
\newcommand\vdj[1]{\left< \Delta J \right>_{#1}}
\newcommand{\w}{{\scriptstyle\wedge}\!}
\newcommand{\Z}{\mathbb{Z}}

\newcommand{\dbar}[1]{\ensuremath{\mathchar'26\mkern-12mu \mathrm{d}^{#1}}\!}


%--------------------------- New Greek
\newcommand{\tht}{\ensuremath{\theta}}
\newcommand{\bet}{\ensuremath{\bar{\eta}}}
\newcommand{\bps}{\ensuremath{\bar{\psi}}}
\newcommand{\bc}{\ensuremath{\bar{\chi}}}
\newcommand{\Bps}{\ensuremath{\bar{\Psi}}}
\newcommand{\Bx}{\ensuremath{\bar{\Xi}}}
\newcommand{\bph}{\ensuremath{\bar{\phi}}}
\newcommand{\vph}{\ensuremath{\varphi}}
\newcommand{\bvph}{\ensuremath{\bar{\varphi}}}
\newcommand{\bth}{\ensuremath{\bar{\theta}}}
\newcommand{\hph}{\ensuremath{\hat{\phi}}}

\newcommand{\bs}[1]{\boldsymbol{#1}}


\renewcommand{\div}{{\mathbf{div}}}
\newcommand{\grad}{{\mathbf{grad}}}
\newcommand{\curl}{{\mathbf{curl}}}

\newcommand\VI[2]{\,\hat{e}^{\hat{#1}}_{\hat{#2}}}
\newcommand\VIF[1]{\,\hat{\boldsymbol{e}}^{\hat{#1}}}
\newcommand\VIN[2]{\;\hat{E}^{\hat{#1}}_{\hat{#2}}}
\newcommand\VINF[1]{\,\hat{\mathbf{E}}_{\hat{#1}}}
\newcommand\hvi[2]{\;\hat{e}^{{#1}}_{{#2}}}
\newcommand\hvin[2]{\;\hat{E}^{{#1}}_{{#2}}}
\newcommand\hvif[1]{\,\hat{\mathbf{e}}^{{#1}}}
\newcommand\hvinf[1]{\,\hat{\mathbf{E}}_{{#1}}}
\newcommand\vi[2]{e^{{#1}}_{{#2}}}
\newcommand\vin[2]{E^{{#1}}_{{#2}}}
\newcommand\vif[1]{\,{\mathbf{e}}^{{#1}}}
\newcommand\vinf[1]{\,{\mathbf{E}}_{{#1}}}
%% \newcommand\Vi[2]{e^{\hat{#1}}_{\hat{#2}}}
%% \newcommand\Vin[2]{E^{\hat{#1}}_{\hat{#2}}}
\newcommand\GAM[1]{{\gamma}^{\hat{#1}}}
%% \newcommand\Gam[1]{\gamma^{\hat{#1}}} 
\newcommand\gam[1]{\gamma^{{#1}}}
\newcommand\hgam[1]{\hat{\gamma}^{{#1}}}
\newcommand\NAB[1]{\hat{\nabla}_{\hat{#1}}}
\newcommand\Nab[1]{\nabla_{\hat{#1}}}
\newcommand\nab[1]{\nabla_{{#1}}}
\newcommand\PA[1]{\partial_{\hat{#1}}}
\newcommand\pa[1]{\partial_{{#1}}}
\newcommand\PAU[1]{\partial^{\hat{#1}}}
\newcommand\pau[1]{\partial^{{#1}}}
\newcommand\lf[1]{{\omega}^{{#1}}}
\newcommand\lft[1]{\hat{\omega}^{{#1}}}
\newcommand\SPI[1]{\;\hat{\omega}_{\hat{#1}}}
\newcommand\SPIF[2]{\,\hat{\boldsymbol{\omega}}^{\hat{#1}}{}_{\hat{#2}}}
%% \newcommand\Spi[1]{\omega_{\hat{#1}}}
\newcommand\spi[1]{\omega_{{#1}}}
\newcommand\tspi[1]{\tilde{\omega}_{{#1}}}
\newcommand\spif[2]{\,{\boldsymbol{\omega}}^{{#1}}{}_{{#2}}}
\newcommand\tspif[2]{\,{\tilde{\boldsymbol{\omega}}}^{{#1}}{}_{{#2}}}
\newcommand\hspi[1]{\hat{\omega}_{{#1}}}
\newcommand\hspif[2]{\,\hat{\boldsymbol{\omega}}^{{#1}}{}_{{#2}}}
%%%%%%%%% Beware of the inconsistency between
%%%%%%%%% \Rif and \RIF
\newcommand{\RIF}[2]{\,\hat{\boldsymbol{\mathcal{R}}}^{\hat{#1}}{}_{\hat{#2}}}
\newcommand{\hRif}[2]{\,\hat{\boldsymbol{\mathcal{R}}}^{{#1}}{}_{{#2}}}
\newcommand{\Rif}[2]{\,\boldsymbol{\mathcal{R}}^{{#1}}{}_{{#2}}}
\newcommand{\tRif}[2]{\,\tilde{\boldsymbol{\mathcal{R}}}^{{#1}}{}_{{#2}}}
\newcommand{\Tf}[1]{\,\boldsymbol{\mathcal{T}}^{#1}}
\newcommand{\TF}[1]{\,\hat{\boldsymbol{\mathcal{T}}}^{\hat{#1}}}
\newcommand{\Tor}[2]{\mathcal{T}^{#1}{}_{#2}}
\newcommand{\cont}[3]{\mathcal{K}_{#1}{}^{#2}{}_{#3}}
\newcommand{\contf}[2]{\,\boldsymbol{\mathcal{K}}^{#1}{}_{#2}}
\newcommand{\hcont}[3]{\hat{\mathcal{K}}_{#1}{}^{#2}{}_{#3}}
\newcommand{\hcontf}[2]{\,\hat{\boldsymbol{\mathcal{K}}}^{#1}{}_{#2}}
\newcommand{\CONT}[3]{\hat{\mathcal{K}}_{\hat{#1}}{}^{\hat{#2}}{}_{\hat{#3}}}
\newcommand{\CONTF}[2]{\,\hat{\boldsymbol{\mathcal{K}}}^{\hat{#1}}{}_{\hat{#2}}}

%% \newcommand{\ket}[1]{\left.\left|#1\right.\right>}
%% \newcommand{\bra}[1]{\left.\left<#1\right.\right|}
\renewcommand\bra[1]{\Bra{#1}}
\renewcommand\ket[1]{\Ket{#1}}
\newcommand{\bkt}[3]{\Braket{ {#1} | {#2} | {#3} } }
\newcommand{\bk}[2]{\Braket{ {#1} | {#2} } }
\newcommand{\comm}[2]{\left[#1,#2\right]}
\newcommand{\acomm}[2]{\left\{#1,#2\right\}}
\newcommand{\vev}[1]{\ensuremath{\left<#1\right>}}
\renewcommand{\set}[1]{\ensuremath{\Set{ #1 }}}

\newcommand{\relphantom}[1]{\mathrel{\phantom{#1}}}

\newcommand{\Ric}{\operatorname{Ric}}
\newcommand*{\diag}{\operatorname{diag}}
\newcommand{\id}{\operatorname{id}}
\newcommand{\tr}{\operatorname{tr}}
\newcommand{\Tr}{\operatorname{Tr}}
\newcommand{\Ker}{\operatorname{Ker}}
\renewcommand{\Im}{\operatorname{Im}}
\newcommand{\sgn}{\operatorname{sgn}}
\newcommand{\Ln}{\operatorname{Ln}}
\newcommand{\Ei}{\operatorname{Ei}}
\newcommand{\csch}{\operatorname{csch}}
\newcommand{\arcsinh}{\operatorname{arcsinh}}
\DeclareMathOperator\Br{Br}

\newcommand{\beq}{\begin{equation}}
\newcommand{\eeq}{\end{equation}}
\newcommand{\ber}{\begin{eqnarray}}
\newcommand{\eer}{\end{eqnarray}}

\renewcommand{\(}{\left(}
\renewcommand{\)}{\right)}
\renewcommand{\[}{\left[}
\renewcommand{\]}{\right]}

\newcommand{\uf}[2][]{\ensuremath{u_{#1}\(\vec{#2}\)}}
\newcommand{\ufb}[2][]{\ensuremath{\bar{u}_{#1}\(\vec{#2}\)}}
\newcommand{\vf}[2][]{\ensuremath{v_{#1}\(\vec{#2}\)}}
\newcommand{\vfb}[2][]{\ensuremath{\bar{v}_{#1}\(\vec{#2}\)}}
\newcommand\pol[2][]{\ensuremath{\varepsilon_{#1}(\vec{#2})}}
\newcommand\polc[2][]{\ensuremath{\varepsilon^*_{#1}(\vec{#2})}}
\newcommand{\ann}[3]{\ensuremath{#1\(\vec{#2},#3\)}}
\newcommand{\cre}[3]{\ensuremath{#1^\dag\(\vec{#2},#3\)}}
%% \newcommand{\uf}[2]{\ensuremath{u\(\vec{#1},#2\)}}
%% \newcommand{\ufb}[2]{\ensuremath{\bar{u}\(\vec{#1},#2\)}}
%% \newcommand{\vf}[2]{\ensuremath{v\(\vec{#1},#2\)}}
%% \newcommand{\vfb}[2]{\ensuremath{\bar{v}\(\vec{#1},#2\)}}
%% \newcommand{\ann}[3]{\ensuremath{#1\(\vec{#2},#3\)}}
%% \newcommand{\cre}[3]{\ensuremath{#1^\dag\(\vec{#2},#3\)}}

\newcommand{\dif}{{\mathrm{d}}}
\newcommand{\difn}[1]{{\mathrm{d}}^{#1}}
\newcommand{\dn}[2]{\,{\mathrm{d}}^{#1}\!{#2}\;}
\newcommand{\dbarn}[2]{\,\dbar{#1}{#2}\;}
\newcommand*{\de}[1]{\mathop{\mathrm{d}#1}\nolimits}% differential, facultative argoment between square parentheses
\newcommand*{\desec}[1][]{\mathop{\mathrm{d^2}#1}\nolimits}% second differential, facultative argoment between square parentheses
\newcommand{\der}[2]{\frac{\de{#1}}{\de{#2}}}% first derivative 
%\newcommand{\pder}[2]{\frac{\pa{}#1}{\pa{}{#2}}}% first derivative 
\newcommand{\inlineder}[2]{\mathrm{d}{#1}/\mathrm{d}{#2}}% in-line first derivative
\newcommand{\dersec}[2]{\frac{{\desec[#1]}}{\de[{#2}^2]}}% second derivative
%% \newcommand{\dx}{\de[x]}% frequently used differentials
%% \newcommand{\dy}{\de[y]}
%\newcommand{\df}{\de[f]}

%------------------
%--------- Format
%------------------
\newcommand{\out}[1]{{\color{red} \sout{#1}}}
\newcommand{\pro}[1]{{\color{blue!70!black} #1}}




\hypersetup{%
  pdftitle={Purely Affine Gravity},
  pdfauthor={Aureliano Skirzewski,}{Oscar Castillo-Felisola},
  pdfkeywords={Affine Gravity,} {Torsion,} {Generalised Gravity.},
  pdflang={English}
}



%------------------
%--------- Document
%------------------
\begin{document}

\title{Purely affine gravity}


\author{Aureliano \surname{Skirzewski}}
\email[Corresponding Author: ]{skirz@ula.ve}
\affiliation{Centro de F\'\i sica Fundamental,  Universidad de los Andes, 5101 M\'erida, Venezuela.}

\author{Oscar \surname{Castillo-Felisola}}
\email{o.castillo.felisola@gmail.com}
\affiliation{Centro Cient\'\i fico Tecnol\'ogico de Valpara\'\i so, Chile.}
\affiliation{Departamento de F\'\i sica, Universidad T\'{e}cnica Federico Santa Mar\'\i a, Casilla 110-V, Valpara\'\i so, Chile.}

%--------- Abstract
\begin{abstract}
  We develop a topological theory of gravity with torsion where metric has a dynamical rather than a kinematical origin. This approach towards gravity resembles pre-geometrical approaches in which a fundamental metric is not assumed even though the  affine connection gives place to a local inertial structure, which reminds us of Mach's principle that assumes the inertial forces, subject to locally Minkowskian metric structures, should have dynamical origin. Additionally a Newtonian like gravitational force is obtained in the perturbative limit of the theory.
\end{abstract}

\pacs{04.50.-h,04.62.+v,11.25.Mj,11.25.-w}
\keywords{Affine Gravity, Torsion, Generalised Gravity.}

\maketitle

\section{Introduction}


Mach proposed that inertial forces should have a dynamical rather than a kinematical origin. Suppose for a second the possibility that there were no stars to be seen and no reference with respect to which an astronaut on space, simply floating either rotating or not, could say that he is rotating. Then, one cannot argue that he is rotating unless inertial forces appear.  The question arises, Should inertial forces  be in correspondence with the presence of matter elsewhere in the universe?  The attention of the reader is called to the fact that any locally Minkowskian metric in the kinematics of the description of spacetime will introduce a notion of inertial forces at a microscopic level. One will therefore explore the dynamical origin of inertial forces, by studying the appearance of a spacetime metric from a renormalizable model to describe the evolution of the affine connection of a manifold with torsion. In this model, it has been constructed the most general action that might be power counting renormalizable that includes only the 64 components gauge connection associated with diffeomorphisms invariance.

General Relativity (GR) has proven to be the most successful theory of gravity.  However,  it is not as successful as we may wish. Part of the problem with GR is that the standard quantization procedure cannot be applied  properly on it. Moreover, not only it is not renormalizable, but there are  problems with the choice of variables to be quantized and the choice of the Hilbert space to be used. 

Although there is nothing wrong about metric spacetimes,  certainly it is an issue to sum over all possible field configurations of the metric, as this would imply summing Euclidean and Minkowski like contributions to the transition amplitudes on equal terms. Additionally, we might also consider the difficulties of  quantizing  non polynomial functions of the three-dimensional metric that appears in the Hamiltonian in an ADM formulation of GR, such as the square root of the metric's determinant or the Ricci scalar.

In order to address some of these issues, several approaches have been designed that use the connection as a fundamental field. For instance, in the context of Cartan formulations of gravity, letting the metric background become flat we can rewrite Einstein-Hilbert's Lagrangian as a function of the torsion field. This approach is known as Teleparallel Gravity and is equivalent to GR in spite of taking torsion as the fundamental gravitational field.

Furthermore, another alternative description of GR developed initially by Ashtekar uses the spin connection as the fundamental field and the frame field turns out to be its canonically conjugated momentum. In the context of Loop Quantum Gravity (LQG), using Ashtekar variables, a successful quantization program has been achieved. Some  strengths of this quantization program lies within a theorem by Hanno Sahlman that states the only diffeomorphisms invariant Hilbert space that supports the Heisenberg algebra, for the connection and its associated momentum, is the one of Loop Quantum Gravity. In spite of its success, LQG has not advanced enough to conclude that its low energy effective description is GR. Currently, there is no clue about the LQG effective description at other scales,  nor its continuum spacetime limit either. Therefore, we cannot conclude that the search for a fundamental theory of gravitational interactions has ended. On the contrary, there are many alternatives to the usual metric description of Gravity and they all must be tested against experiments and observations.

In this article we study a diffeomorphism invariant ``toy'' model  consisting  solely of an affine connection  (with torsion), and the strict commitment to formulate a power counting renormalizable action.  We find the condition on the background metric to be exactly zero to be appealing because the use of a background metric such as in teleparallel theories of gravity would in principle break background independence. We expect  this strategy may serve to overcome  the uniqueness theorem about diffeomorphism invariant theories of connections, since we  have no fundamental metric field to quantize. The earliest model (that we know of) that argues  a description of gravitational interaction in terms of connections as fundamental fields  was presented by Eddington~\cite{Eddington1923math}, for an spacetime with positive cosmological constant. He proposed the square root of the determinant of the Ricci tensor as the gravitational Lagrangian. His aim was not to solve the problems of quantum gravity,  others have emphasized the character of GR as a gauge theory in order to address the issues of quantization and regularization, as in LQG.
%% Among other, the work by  Krasnov studies a BF model that is conceived to {\bfseries \color{blue} TO BE CONTINUED...}{\color{red}no se que decir al respecto}

The aim of this work is to describe certain aspects of the gravitational interaction in four dimensions. To this end we have studied the non-relativistic limit of the parallel transport equations of motion of a particle moving on a flat, static, homogeneous and isotropic background. Additionally, a  symmetric tensor density which acts like an ``inverse metric'' can be obtained by identifying
%% is a composite field defined in terms of the connection, \hl{Is it still TRUE?}
\begin{equation}\label{metric}
  \sqrt{g}g^{\mu\nu} = \frac{\delta\ }{\delta R_{(\mu\nu)}} S[\Gamma]
\end{equation}
where $g$ is the inverse of the determinant of $g^{\mu\nu}$.

The structure of the paper ...

%A gravitational theory can still be formulated in this non-metric spacetimes, and are known as \textit{affine gravity}.



%--------- Scattered Ideas
\section{Warming up: The three-dimensional case}

Formally, the curvature of a manifold is defined through the commutator of covariant derivatives under diffeomorphims, $\nabla_\mu$, but for general choice of the connection, $\Ga^\mu{}_{\nu\lambda}$, there is an extra contribution given by its antisymmetric part in the lower indices, $T^\mu{}_{\nu\lambda}=2\Ga^\mu{}_{[\nu\lambda]}$. Therefore, the commutator of the covariant derivatives yields,
\begin{equation}
  \comm{\nab{\mu}}{\nab{\nu}}V^\rho = R_{\mu\nu}{}^\rho{}_\lambda V^\lambda - T^\rho{}_{\mu\nu}\nab{\rho}V^\rho.
  \label{curvdef}
\end{equation}
Note that $T^\rho{}_{\mu\nu}$ is a $\tfrac{d^2(d-1)}{2}$ dimensional tensor representation under diffeomorphisms.

In order to build topological invariants of density one, one can use the skew-symmetric Levi-\v{C}ivita tensor $\epsilon^{\mu_1\mu_2\dots\mu_n}$ in $n$-dimensional space(-time).

\begin{widetext}
  With these ingredients, in a three-dimensional space one can write an action 
  \begin{dmath}
    \label{accion3d}
    %\begin{split}      
    S[\Gamma] =
    \int \dn{3}{x}  \Bigg\{R_{\mu_1\mu_2}{}^\rho{}_{\mu_3} T^\sigma{}_{\mu_4\mu_5}\sum_{\pi \in  \mathrm{Z}_5}C_\pi\delta_\rho^{\mu_{\pi(1)}} \delta_\sigma^{\mu_{\pi(2)}}\epsilon^{\mu_{\pi(3)}\mu_{\pi(4)}\mu_{\pi(5)}}  + T^\rho{}_{\mu_1\mu_2} T^\sigma{}_{\mu_3\mu_4}T^\tau{}_{\mu_5\mu_6} \sum_{\pi \in \mathrm{Z}_6}D_\pi\delta_\rho^{\mu_{\pi(1)}} \delta_\sigma^{\mu_{\pi(2)}}\delta_\tau^{\mu_{\pi(3)}}\epsilon^{\mu_{\pi(4)}\mu_{\pi(5)}\mu_{\pi(6)}} + T^\rho{}_{\mu_1\mu_2}\nabla_{\mu_3} T^\sigma{}_{\mu_4\mu_5}\sum_{\pi \in \mathrm Z_5}E_\pi\delta_\rho^{\mu_{\pi(1)}} \delta_\sigma^{\mu_{\pi(2)}}\epsilon^{\mu_{\pi(3)}\mu_{\pi(4)}\mu_{\pi(5)}} \Bigg\}, 
    %\end{split}
  \end{dmath}
  where all possible permutations of $n$ elements $\pi\in\mathrm{Z}_n$ have been included in the sums with  different constants $C_\pi$, $D_\pi$ and $E_\pi$ for  permutation. 
\end{widetext}

The torsion field can be decomposed into invariant tensors respecting the symmetry,
\begin{equation}
  T^\sigma{}_{\mu\nu}=\epsilon_{\mu\nu\rho} T^{\sigma\rho}+A_{[\mu}\delta^\sigma{}_{\nu]},
\end{equation}
with a symmetric $T^{\sigma\rho}$ of density weight  $w =1$, and \mbox{$A_\mu=T^\nu{}_{\mu\nu}$} is the trace part of the more arbitrary $T^\sigma{}_{\mu\nu}$.

The action in Eq.~\eqref{accion3d} can be rewritten as
\begin{dmath}[compact, spread=2pt]
  \label{accion3dnew}
  S[\Gamma]=\int\dn{3}{x} \Bigg( 
  B_1R_{\mu\nu}{}^\mu{}_{\rho} T^{\nu\rho} 
  +B_2\epsilon^{\mu\nu\rho}R_{\mu\nu}{}^{\sigma}{}_\sigma A_\rho 
  +B_3 \epsilon^{\mu\nu\rho}R_{\mu\nu}{}^\lambda{}_\rho A_\lambda 
  +B_4 \det(T^{\mu\nu}) 
  +B_5 T^{\mu\nu}A_\mu A_\nu 
  +B_6 T^{\mu\nu}\nabla_\mu A_\nu
  +B_7\epsilon^{\mu\nu\rho}A_\mu\partial_\nu A_\rho
  + B_8\epsilon^{\mu\nu\rho}\Gamma^{\sigma}{}_{\mu\sigma}\partial_\nu\Gamma^{\tau}{}_{\rho\tau}
  +B_9\epsilon^{\mu\nu\lambda}\Big(\Gamma^{\sigma}{}_{\mu\rho}\partial_\nu\Gamma^{\rho}{}_{\lambda\sigma}
  +\frac{2}{3}\Gamma^{\tau}{}_{\mu\rho}\Gamma^{\rho}{}_{\nu\sigma}\Gamma^{\sigma}{}_{\lambda\tau}{}\Big)
  \Bigg).
\end{dmath}
where the coefficients $B_i$ are related with the original coefficients $C_i$, $D_i$ and $E_i$, and the additional $B_9$ term can be added in three dimensions leaving the action invariant under diffeomorphisms. The affine connection can be decomposed into its symmetric and antisymmetric parts, 
\begin{equation}
  \Gamma^\lambda{}_{\mu\rho}=\hat{\Gamma}^\lambda{}_{(\mu\rho)} + \epsilon_{\mu\rho\sigma}T^{\lambda\sigma} + A_{[\mu}\delta^\lambda{}_{\rho]},
\end{equation}
where  $\epsilon_{\mu\rho\sigma}$ has been introduced, and it is related to the skew symmetric $\epsilon^{\mu\rho\sigma}$ through the identity \mbox{$\epsilon^{\lambda\mu\nu}\epsilon_{\rho\sigma\tau}=3!\delta^{\lambda}{}_{[\rho}\delta^\mu{}_{\sigma}\delta^{\nu}{}_{\tau]}$.} Therefore, the curvature tensor can be expressed as 
\begin{dmath}[compact, spread=2pt]
  \label{RiemmanDecomposition}
  R_{\mu\nu}{}^\sigma{}_\rho=
  \hat{R}_{\mu\nu}{}^\sigma{}_\rho
  -2\epsilon_{\rho\alpha[\mu}\hat\nabla_{\nu]}T^{\sigma\alpha}
  +\partial_{[\mu}A_{\nu]}\delta^\sigma_\rho
  +\delta^\sigma_{[\mu}\hat\nabla_{\nu]}A_\rho
  +\epsilon_{\mu\nu\kappa}T^{\kappa\sigma}A_\rho
  -\delta^\sigma_{[\mu}\epsilon_{\nu]\rho\alpha}T^{\alpha\beta}A_\beta 
  +\frac{1}{2}\delta^\sigma_{[\mu}A_{\nu]}A_\rho
  -2\epsilon_{\alpha\beta[\mu}\epsilon_{\nu]\rho\delta}T^{\sigma\alpha}T^{\beta\delta},
\end{dmath}
where $ \hat\nabla_\rho$ and $\hat{R}_{\mu\nu}{}^\lambda{}_\rho$ are the covariant derivative and  curvature associated to the symmetric part of the connection. Notice that Bianchi identity, obtained as $\epsilon^{\mu\nu\lambda}\hat R_{\mu\nu}{}^\rho{}_\lambda=0$, leads us to the following
\begin{equation}
  \label{bianchi}
  \epsilon^{\mu\nu\rho} R_{\mu\nu}{}^\lambda{}_\rho = 4\hat\nabla_\rho T^{\rho\lambda}
  +2\epsilon^{\mu\nu\lambda}\partial_\mu A_\nu-4T^{\lambda\rho}A_\rho. 
\end{equation}
Using the  Eqs.~\eqref{RiemmanDecomposition} and \eqref{bianchi} one can rewrite the action in Eq.~\eqref{accion3dnew} as
\begin{dmath}[compact, spread=2pt]
  \label{accion3dfinal} 
  S[\Gamma] = \int\dn{3}{x} \Bigg(  
  (B_1+2B_9) \hat R_{\mu\nu}{}^\mu{}_{\rho} T^{\nu\rho}   
  + (B_2+B_9+B_9) \epsilon^{\mu\nu\rho}\hat R_{\mu\nu}{}^{\sigma}{}_\sigma A_\rho  
  + (-6B_1+B_4-4B_9) \det(T^{\mu\nu})   
  + (\frac{1}{2}B_1-4B_3+B_5+B_9 ) T^{\mu\nu}A_\mu A_\nu   
  + (B_1-4B_3+B_6+2B_9) T^{\mu\nu}\hat\nabla_\mu A_\nu  
  + (2B_2+2B_3+B_7+B_9)\epsilon^{\mu\nu\rho}A_\mu\partial_\nu A_\rho  
  + B_9\epsilon^{\mu\nu\lambda}\Big(\hat\Gamma^{\sigma}{}_{\mu\rho}\partial_\nu\hat\Gamma^{\rho}{}_{\lambda\sigma}  + \frac{2}{3}\hat\Gamma^{\tau}{}_{\mu\rho}\hat\Gamma^{\rho}{}_{\nu\sigma}{}\hat\Gamma^{\sigma}{}_{\lambda\tau}{}\Big)  
  + B_8\epsilon^{\mu\nu\rho}\hat\Gamma^{\sigma}{}_{\mu\sigma}\partial_\nu\hat\Gamma^{\tau}{}_{\rho\tau}  
  + \partial_\alpha\Big( 4B_3T^{\alpha\mu} A_\mu +B_9(2\Gamma^{[\delta}{}_{\delta\beta}T^{\alpha]\beta}-T^{\alpha\beta}A_\beta+\frac{1}{2}\epsilon^{\beta\alpha\eta}A_\eta\Gamma^{\delta}{}_{\delta\beta})  +B_8\epsilon^{\beta\alpha\eta}A_\eta\hat\Gamma^{\delta}{}_{\delta\beta}\Big)  \Bigg),
\end{dmath}
or after dropping the boundary terms and rename the coefficients,
\begin{dmath}[compact, spread=2pt]
  S[\hat\Gamma,T,A] =
  \int \dn{3}{x} \bigg( 
  A_1\hat{R}_{\mu\nu}{}^{\mu}{}_\rho T^{\nu\rho} 
  +A_2\epsilon^{\mu\nu\rho}\hat{R}_{\mu\nu}{}^{\sigma}{}_\sigma A_\rho
  +A_3\epsilon^{\mu\nu\rho}A_\mu\partial_\nu A_\rho
  +A_4T^{\mu\nu}\hat{\nabla}_\mu A_\nu
  +A_5T^{\mu\nu}A_\mu A_\nu
  +A_6\det(T^{\mu\nu}) 
  +A_7\epsilon^{\mu\nu\lambda}\Big(\hat{\Gamma}^{\sigma}{}_{\mu\rho}\partial_\nu\hat{\Gamma}^{\rho}{}_{\lambda\sigma}
  +\frac{2}{3}\hat{\Gamma}^{\tau}{}_{\mu\rho}\hat{\Gamma}^{\rho}{}_{\nu\sigma}{}\hat{\Gamma}^{\sigma}{}_{\lambda\tau}{}\Big)
  + A_8\epsilon^{\mu\nu\rho}\hat{\Gamma}^{\sigma}{}_{\mu\sigma}\partial_\nu\hat{\Gamma}^{\tau}{}_{\rho\tau}
  \bigg).
\end{dmath}

%Poplawski...
Noticing that in  the first term, the variation respect to the Ricci tensor yields to $T^{\mu\nu}$, it can be argued that in a standard theory of gravity this tensor density corresponds to $\sqrt{g}g^{\mu\nu}$.
Therefore, Eq.~\eqref{accion3dfinal} reveals a one to one correspondence with General Relativity non minimally coupled to the $A_\mu$ field, %\hl{I'd say that this result should be presented in terms of the action with coefficients C's... This result should be review!!!}{\color{blue} ya?}
\begin{dmath}
  \label{accion3dGR}
  S[g,\hat{\Gamma},A] = \int \dn{3}{x} \bigg(
  \sqrt{g} \Big(  A_1 \hat{R} + A_4\hat{\nabla}^\mu A_\mu + A_5 A_\mu A^\mu + A_6  \Big)
  + A_2\epsilon^{\mu\nu\rho} \hat{R}_{\mu\nu}{}^{\sigma}{}_\sigma A_\rho
  + A_3\epsilon^{\mu\nu\rho}A_\mu\partial_\nu A_\rho
  + A_7\epsilon^{\mu\nu\lambda}\Big(\hat{\Gamma}^{\sigma}{}_{\mu\rho}\partial_\nu\hat{\Gamma}^{\rho}{}_{\lambda\sigma}
  + \frac{2}{3}\hat{\Gamma}^{\tau}{}_{\mu\rho}\hat{\Gamma}^{\rho}{}_{\nu\sigma}{}\hat{\Gamma}^{\sigma}{}_{\lambda\tau}{}\Big)
  + A_8\epsilon^{\mu\nu\rho}\hat{\Gamma}^{\sigma}{}_{\mu\sigma}\partial_\nu\hat{\Gamma}^{\tau}{}_{\rho\tau}
  \bigg)
\end{dmath}
Thus, an interesting sector of the theory corresponds to the space of non-degenerated $T^{\mu\nu}$. 

As a matter of fact, the vector field and the ``gravitational" sector of the torsion disengage when $A_2=0$.


\section{4D metricless (and torsionful) action}

Following the precepts  already learnt, one starts  by defining an irreducible representation decomposition for the full connection field 
\begin{equation}
  \Gamma^\mu{}_{\rho\sigma} = \hat{\Gamma}^\mu{}_{\rho\sigma} + T^\mu{}_{\rho\sigma} = \hat{\Gamma}^\mu{}_{\rho\sigma} + \epsilon_{\rho\sigma\lambda\kappa}T^{\mu,\lambda\kappa}+A_{[\rho}\delta^\mu_{\nu]},
\end{equation}
where $\hat{\Gamma}^\mu{}_{\rho\sigma}$ denotes a $\tfrac{d^2(d+1)}{2}$ dimensional symmetric connection, $A_\mu$ is a $d$ dimensional vector field  that gives trace to the antisymmetric part of the full connection, and  $T^{\mu,\lambda\kappa}$ is a $\tfrac{d(d+1)(d-2)}{2}$ dimensional (Curtright) field that is defined through the symmetry of its indices: antisymmetric in the last two indices, and it has a cyclic property $T^{\mu,\lambda\kappa}+T^{\lambda,\kappa\mu}+T^{\kappa,\mu\lambda}=0$. In other words that $T^{[\mu,\lambda]\kappa}=\frac{1}{2}T^{\kappa,\lambda\mu}$, just as for  the Riemmann tensor $\hat{R}_{\mu[\nu}{}^\alpha{}_{\lambda]}=\frac{1}{2}\hat{R}_{\lambda\nu}{}^\alpha{}_{\mu}$. Notice that due to its symmetries, the contraction $\epsilon_{\rho\sigma\lambda\kappa}T^{\mu,\lambda\kappa}$ is traceless.

Additionally, since no metric is present  the epsilon symbols are not related by lowering or raising their indices, but instead one demands that \mbox{$\epsilon^{\delta\eta\lambda\kappa}\epsilon_{\mu\nu\rho\sigma}=4!\delta^{\delta}{}_{[\mu}\delta^\eta{}_{\nu}\delta^{\lambda}{}_{\rho} \delta^\kappa{}_{\sigma]}$.}

\begin{widetext}
  One can  write all the combinations of fields that would presumably be renormalizable with these three independent fields
  \begin{dmath}[compact, spread=2pt] 
    \label{4dfull}
    S[\hat{\Gamma},T,A] =
    \int\dn{4}{x}\Bigg[\partial_\lambda\bigg(
    A_1\hat{ R}_{\mu(\nu}{}^{\mu}{}_{\rho)} T^{\nu,\rho\lambda} 
    +A_2\epsilon^{\lambda\mu\nu\rho}\hat{ R}_{\mu\nu}{}^{\sigma}{}_\sigma A_\rho
    +A_3\epsilon^{\lambda\mu\nu\rho}A_\mu\partial_\nu A_\rho
    +A_4T^{(\mu,\nu)\lambda}\hat\nabla_\mu A_\nu
    +A_5T^{(\mu,\nu)\lambda}A_\mu A_\nu
    +A_6\epsilon_{\mu\nu\rho\sigma}\epsilon_{\alpha\beta\gamma\delta} T^{\lambda,\mu\alpha} T^{\beta,\rho\sigma} T^{\nu,\gamma\delta} 
    +A_7\epsilon^{\lambda\mu\nu\lambda}\Big(\hat\Gamma^{\sigma}{}_{\mu\rho}\partial_\nu\hat\Gamma^{\rho}{}_{\lambda\sigma}
    +\frac{2}{3}\hat\Gamma^{\tau}{}_{\mu\rho}\hat\Gamma^{\rho}{}_{\nu\sigma}{}\hat\Gamma^{\sigma}{}_{\lambda\tau}{}\Big) 
    + A_8\epsilon^{\lambda\mu\nu\rho}\hat\Gamma^{\sigma}{}_{\mu\sigma}\partial_\nu\hat\Gamma^{\tau}{}_{\rho\tau}
    +A_9 R_{\mu\nu}{}^{\lambda}{}_{\rho} T^{\rho,\mu\nu}
    +A_{10}T^{\lambda,\alpha\beta}T^{\kappa,\gamma\delta} A_\kappa\epsilon_{\alpha\beta\gamma\delta}
    \bigg)
    +B_1 \hat R_{\mu\nu}{}^{\mu}{}_{\rho} T^{\nu,\alpha\beta}T^{\rho,\gamma\delta}\epsilon_{\alpha\beta\gamma\delta}
    +B_2 \Big(\hat R_{\mu\nu}{}^{\sigma}{}_\rho+\frac{2}{3}\delta^\sigma{}_{[\mu}\hat R_{\nu]\lambda}{}^{\lambda}{}_\rho \Big) T^{\beta,\mu\nu}T^{\rho,\gamma\delta}\epsilon_{\sigma\beta\gamma\delta}
    +B_3 \hat R_{\mu\nu}{}^{\mu}{}_{\rho} T^{(\nu,\rho)\sigma}A_\sigma
    + B_4\Big(\hat R_{\mu\nu}{}^{\sigma}{}_\rho+\frac{2}{3}\delta^\sigma{}_{[\mu}\hat R_{\nu]\lambda}{}^{\lambda}{}_\rho \Big)\Big(T^{\rho,\mu\nu}A_\sigma-\frac{1}{4}\delta^\rho_\sigma T^{\kappa,\mu\nu}A_\kappa\Big)
    +B_5\hat R_{\mu\nu}{}^{\rho}{}_\rho T^{\sigma,\mu\nu}A_\sigma
    +C_1 \hat R_{\mu\nu}{}^{\mu}{}_{\rho} \hat\nabla_\sigma T^{(\nu,\rho)\sigma}
    +C_2\Big(\hat R_{\mu\nu}{}^{\sigma}{}_\rho+\frac{2}{3}\delta^\sigma{}_{[\mu}\hat R_{\nu]\lambda}{}^{\lambda}{}_\rho \Big)\Big(\hat\nabla_\sigma T^{\rho,\mu\nu}-\frac{1}{4}\delta^\rho_\sigma \hat\nabla_\kappa T^{\kappa,\mu\nu}\Big)
    +C_3\hat R_{\mu\nu}{}^{\rho}{}_\rho \hat\nabla_\sigma T^{\sigma,\mu\nu} 
    +D_1T^{\alpha,\mu\nu}T^{\beta,\rho\sigma}\hat\nabla_\gamma T^{(\lambda, \kappa) \gamma}\epsilon_{\beta\mu\nu\lambda}\epsilon_{\alpha\rho\sigma\kappa}
    +D_2T^{\alpha,\mu\nu}T^{\lambda,\beta\gamma}\hat\nabla_\lambda T^{\delta,\rho\sigma}\epsilon_{\alpha\beta\gamma\delta}\epsilon_{\mu\nu\rho\sigma}
    +D_3T^{\mu,\alpha\beta}T^{\lambda,\nu\gamma}\hat\nabla_\lambda T^{\delta,\rho\sigma}\epsilon_{\alpha\beta\gamma\delta}\epsilon_{\mu\nu\rho\sigma}
    +D_4T^{\lambda,\mu\nu}T^{\kappa,\rho\sigma}\hat\nabla_{(\lambda} A_{\kappa)} \epsilon_{\mu\nu\rho\sigma}
    +D_5T^{\lambda,\mu\nu}\hat\nabla_{[\lambda}T^{\kappa,\rho\sigma} A_{\kappa]} \epsilon_{\mu\nu\rho\sigma}
    +D_6T^{\lambda,\mu\nu}A_\nu\hat\nabla_{(\lambda} A_{\mu)}
    +D_7T^{\lambda,\mu\nu}A_\lambda\hat\nabla_{[\mu} A_{\nu]} 
    +E_1\hat\nabla_{(\rho} T^{\rho,\mu\nu}\hat\nabla_{\sigma)} T^{\sigma,\lambda\kappa}\epsilon_{\mu\nu\lambda\kappa}
    +E_2\hat\nabla_{(\lambda} T^{\lambda,\mu\nu}\hat\nabla_{\mu)} A_\nu
    +T^{\alpha,\beta\gamma}T^{\delta,\eta\kappa}T^{\lambda,\mu\nu}T^{\rho,\sigma\tau}
    \big(\Lambda_1\epsilon_{\beta\gamma\eta\kappa}\epsilon_{\alpha\rho\mu\nu}\epsilon_{\delta\lambda\sigma\tau}
    +\Lambda_2\epsilon_{\beta\lambda\eta\kappa}\epsilon_{\gamma\rho\mu\nu}\epsilon_{\alpha\delta\sigma\tau}\big) 
    +\Lambda_3 T^{\rho,\alpha\beta}T^{\gamma,\mu\nu}T^{\lambda,\sigma\tau}A_\tau \epsilon_{\alpha\beta\gamma\lambda}\epsilon_{\mu\nu\rho\sigma}
    +\Lambda_4T^{\eta,\alpha\beta}T^{\kappa,\gamma\delta}A_\eta A_\kappa\epsilon_{\alpha\beta\gamma\delta}\Bigg],
  \end{dmath}
  where the $A_i$ series of terms contribute solely to the boundary conditions, and the terms $B_2$, $B_4$ and $C_2$ contain a traceless contribution of the curvature.
\end{widetext}
In this case, the induced  inverse metric density [see Eq.~\eqref{metric}] is 
\begin{dmath}
  \label{4dMetric}
  \bar{g}^{\mu\nu} \equiv \sqrt{-g}g^{\mu\nu} = B_1T^{\mu,\lambda\kappa}T^{\nu,\rho\sigma}\epsilon_{\lambda\kappa\rho\sigma} + B_3T^{(\mu,\nu)\lambda}A_\lambda + C_1\hat{\nabla}_\lambda T^{(\mu,\nu)\lambda}.
\end{dmath}

Let us define the compatible connection $\bar{\Gamma}^{\lambda}_{\mu\nu}$ (associated to $\bar g^{\mu\nu}$ through $\bar{\nabla}_\lambda \bar{ g}^{\mu\nu}=0$). In $d=4$, (\ref{4dMetric}) could be argued to be the metric if  $\Gamma^\lambda{}_{\mu\nu}=\bar\Gamma^\lambda{}_{\mu\nu}
+{\mathcal K}^\lambda{}_{\mu\nu}$, where ${\mathcal K}^\lambda{}_{\mu\nu}$ is the Contorsion tensor, or equivalently in our variables
\begin{dmath}\label{MetricityCondition}
\Big(\hat\Gamma^{\tau}{}_{\mu\nu}-\bar\Gamma^{\tau}{}_{\mu\nu}\Big)\bar g^{\mu\rho}\bar g^{\nu\sigma}=
-T^{(\rho}{}_{\mu\nu}\bar g^{\sigma)\mu}\bar g^{\nu\tau}	.
\end{dmath} 
If satisfied, this condition would confirm our supposition that what couples to the Ricci is the metric. On the other hand, there could be large sectors of the space of solutions that would not satisfy this requirement but we are specifically tilted towards pseudo-Riemannian spacetimes. In particular, Eq.~\eqref{MetricityCondition} can be solved perturbatively taking care to solve the equations of motion as well.



\subsection*{Symmetry breaking and Equations of motion }

In four dimensions there is no obvious equivalence of Eq.~\eqref{4dfull} with GR, specially due to the lack of a fundamental metric field in the given model. However, both models are explicitly invariant under diffeomorphisms, and even if their structures and number of degrees of freedom differ, the action in Eq.~\eqref{4dfull} provides a context where parallel transport of particle velocities on a purely torsional background is nontrivial.

Here, we will stablish the model's non-relativistic (Newtonian) limit for the ``geodesic" deviation of ``inertial" observers at rest with respect to a static, isotropic, homogeneous and spatially flat background. In order to properly analyse the model, we  propose the following decomposition of the fields
\begin{dmath}
  A_\mu = \delta_\mu^0 A_0 + \delta_\mu^i A_i 
\end{dmath}
and
\begin{dmath}
  T^{\mu,\nu\rho} = \delta^{\mu}_i\delta^{\nu\rho}_{jk}T^{i,jk} + (\delta^{\mu}_0\delta^{\nu\rho}_{ij}-\delta^{\mu}_i\delta^{\nu\rho}_{j0})a^{[ij]} + \delta^{\mu}_i\delta^{\nu\rho}_{j0}T^{(ij)} + \delta^{\mu}_0\delta^{\nu\rho}_{i0}c^i,
\end{dmath}
where $\delta^{\mu\nu}_{\lambda\kappa}=\delta^{\mu}_{\lambda}\delta^{\nu}_{\kappa}-\delta^{\mu}_{\kappa}\delta^{\nu}_{\lambda}$, $T^{i,jk}\epsilon_{ijk}=0$, $a^{ij}$ is antisymmetric, $T^{ij}$ is a symmetric tensor and $c^i$ an arbitrary vector.

In order to make perturbation theory we will expand around a static, isotropic and homogeneous solution of the equations of motion, because these are characteristic of the observable universe. To make sure that it is a solution of the equations of motion, we can expand the action \eqref{4dfull} around it $A_\mu=\delta_\mu^0 A$, $T^{\mu,\nu\rho}=\delta^{\mu}_m\delta^{\nu\rho}_{m0}T$ and $\hat\Gamma^\lambda{}_{\mu\nu}=E\delta^\lambda_0\delta^a_\mu\delta^a_\nu +F\delta^\lambda_a \delta^a_{(\mu}\delta^0_{\nu)}+G\delta^\lambda_0 \delta^0_{\mu}\delta^0_{\nu}$ with the small perturbations $a_\mu$, $t^{\mu,\nu\rho}$, and $\gamma^\lambda{}_{\mu\nu}$ and fix $A,\ T,\ E\ ,F,$ and $G$ such that the action is extreme, i.e. there must be no contribution due to first order perturbations.

Particularly interesting is the induced metric \eqref{4dMetric}, that evaluated on the background is
\begin{dmath}
  \label{3+1metric}
  \sqrt{-g}g^{\mu\nu}=(B_3A+\frac{1}{2}C_1F)T\delta^\mu_m\delta^\nu_m-3C_1ET\delta^\mu_0\delta^\nu_0
\end{dmath}
while the Ricci curvature tensor is
\begin{dmath}
  R_{\mu\nu}=\frac{1}{2}EF\delta^m_\mu\delta^m_\nu-\frac{3}{4}F^2\delta^0_\mu\delta^0_\nu.
\end{dmath}
Thus, whether the four-dimensional metric structure is Riemannian or pseudo-Riemannian will depend exclusively on the values of the parameters of the action in Eq.~\eqref{4dfull} and the signs of the components of the connection field. 
\begin{widetext}
  \begin{dmath}[compact, spread=2pt]
    \label{EOM0thOrder}
    \delta S= \Big( \Big( ( B_3 + \frac{8}{3}\, B_4 + \frac{1}{2}\, E_2) A + (4\, C_1 - \frac{16}{3}\, C_2) F + ( - 2\, C_1 + \frac{8}{3}\, C_2) G \Big) E + 8\, ( - D_1 + 2\, D_2 + D_3) T T \Big) T \delta{\Gamma}^{m}\,_{0 m} + \Big( ( \frac{1}{2}\, B_3 + \frac{4}{3}\, B_4 + \frac{1}{4}\, E_2) A F + ( B_3 - \frac{4}{3}\, B_4 - \frac{1}{2}\, E_2) A G + (C_1 - \frac{4}{3}\, C_2) F F + ( - C_1 + \frac{4}{3}\, C_2) F G - D_6 A A \Big) T \delta{\Gamma}^{0 m}\,_{m} + \Big( \Big(- (\frac{1}{2}\, B_3 + \frac{4}{3}\, B_4 + \frac{1}{4}\, E_2) A F + ( - B_3+ \frac{4}{3}\, B_4 + \frac{1}{2}\, E_2) A G + ( - C_1 + \frac{4}{3}\, C_2) F F + (C_1 - \frac{4}{3}\, C_2) F G + D_6 A A \Big) E +\Big( 12\, ( D_1 - 2\, D_2 - D_3) F + 24\, L_3 A \Big) T T \Big)\delta{T}_{m}\,^{0 m} + \Big( ( 3\, B_3 - 4\, B_4 - \frac{3}{2}\, E_2) A + ( - 3\, C_1 + 4\, C_2) F \Big) E T \delta{\Gamma}^{0}\,_{0 0} + \Big( 3\Big( - 2\, D_6 A + ( \frac{1}{2}\, B_3 + \frac{4}{3}\, B_4 + \frac{1}{4}\, E_2) F + ( B_3 - \frac{4}{3}\, B_4 - \frac{1}{2}\, E_2) G \Big) E - 24\, L_3 T T \Big) T \delta{A}_{0}=0,
  \end{dmath}
\end{widetext}
which can be solved in many ways.
%for example,  $L_3=0$, $C_2=\frac{3}{4}C_1$, $D_3=D_1-2D_2$ and $A=0$, and $G=\frac{6B_3+16B_4+3E_2}{-12B_3+16B_4+6E_2}F$. 
However, we are interested in solutions that contain the maximum symmetry (because symmetries often protect the solutions from changes with conserved quantities). We are going to use  $D_3=D_1-2D_2$, $D_6=0$, $E_2=2 B_3 -\frac{8}{3} B_4$, $L_3=0$, $F=0$, and $G=\frac{B_3+\frac{4}{3}B_4}{C_1 - \frac{4}{3}C_2}A=\sigma A$, because this particular choice solves \eqref{EOM0thOrder} while setting Ricci curvature tensor to zero, which leads to a global $SO(3,1)$ symmetry on the induced metric \eqref{3+1metric}.


%% In order to make perturbation theory we will expand around an isotropic and homogeneous solution, as these are characteristic of the observable universe. Thus, in order to find such solution we can substitute $A_\mu=\delta_\mu^0 A_0$ and $T^{\mu,\nu\rho}=\delta^{\mu}_i\delta^{\nu\rho}_{j0}T^{ij}$; for a spacially flat metric type, with $k=0$, we would choose $T^{ij}=\delta^{ij}T(t)$. For this particular choice we wish to remark that the contraction of torsion terms
%% \begin{align}
%%   T_{\mu}{}^{\sigma}{}_\rho T_{\nu}{}^{\rho}{}_\sigma &= (\epsilon_{\mu\rho\lambda\kappa}T^{\sigma,\lambda\kappa}+A_{[\mu}\delta^\sigma_{\rho]})(\epsilon_{\nu\sigma\delta\eta}T^{\rho,\delta\eta}+A_{[\nu}\delta^\rho_{\sigma]}) \notag \\
%%   &= \delta^0_\mu\delta^0_\nu\frac{3}{4} A_0^2-24\det(T^{ij})\delta^i_\mu\delta^j_\nu T_{ij},
%% \end{align}
%% where $T_{ij}$ is defined as the inverse of $T^{ij}$. Being a covariant tensor, $T_{\mu}{}^{\sigma}{}_\rho T_{\nu}{}^{\rho}{}_\sigma$ can play the role of the metric of the {\it spacetime}, defined as a constructed field $g_{\mu\nu}:= T_{\mu}{}^{\sigma}{}_\rho T_{\nu}{}^{\rho}{}_\sigma $. Similarly, we can define another metric-like tensor, 
%% \begin{equation}
%%   \tilde g^{\lambda\kappa}:= T_{\mu}{}^{\lambda}{}_\nu T_{\rho}{}^{\kappa}{}_\sigma\epsilon^{\mu\nu\rho\sigma}=-8\delta^\lambda_i\delta^\kappa_jA_0T^{ij},
%% \end{equation}
%% but we can see that this would not be a four dimensional metric but only the space part of it.

%% Anyhow, we can now do perturbation theory expanding up to second order the fields around the solution $T^{\mu,\nu\lambda} = \delta^{\mu}_i \delta_{i0}^{\nu\lambda}T + t^{\mu,\nu\lambda}$ and $A_\mu=\delta_\mu^0A_0+a_\mu$ and in the simplest case, where $A_0$ and $T$ are constants in time we can also set $\hat{\Gamma}_\mu{}^\lambda{}_\nu=\gamma_\mu{}^\lambda{}_\nu$, where the lowcase fields are small perturbations. The expansion is required to be performed around a minimum of the action, therefore we will set linear terms to zero. 
%% \begin{equation}
%%   \label{3+1fullPerturbed}
%%    S_0[\hat{\Gamma}_{\mu}{}^{\nu}{}_\rho,T^{\lambda,\mu\nu},A_\mu]=\int\dn{4}{x}\Bigg(-3!A_0T^3\Lambda_3\Bigg),
%% \end{equation}

%% \begin{equation}
%%   \begin{split}
%%     \label{3+1full1stOrder}
%%     S_1[\hat{\Gamma}_{\mu}{}^{\nu}{}_\rho,T^{\lambda,\mu\nu},A_\mu] = \int\dn{4}{x} &\Bigg(2(-4B_2+B_3)T^2\partial_{[i}\gamma_{0]}{}^k{}_j\epsilon_{0ijk} \\
%%     & + 2 A_0 T \big((B_4-B_6)\partial_{[i}\gamma_{0]}{}^0{}_i+B_6\partial_{[i}\gamma_{j]}{}^j{}_i\big)\\
%%     & + 8 T^3 \big(-3D_1\gamma_0{}^0{}_0+(-2D_1+D_2+D_3)\gamma_i{}^i{}_0\big)\\
%%     & - T A_0^2(D_6+D_7)\gamma_i{}^0{}_i -12\Lambda_3( 2t^{k,k0}+a_0)\Bigg).
%%   \end{split}
%% \end{equation}
%% Therefore, we must set $B_3=4B_2$, $B_4=0=B_6$, $D_1=0$, $D_3=-D_2$,  $D_7=-D_6$ and $\Lambda_3=0$ in order to be sure that the homogeneous and isotropic field configuration minimizes the action. 

%% The second order perturbative expansion  is



\nocite{Tucker:1996sx,Horava:2009uw,Lu:2009em,Gibbs:1995gj,WheelerPre,Peeters:2007wn,peeters2007symbolic,Peeters2007550,sage}

\bibliographystyle{aipauth4-1}
\bibliography{References.bib}

\end{document}

%% In the case that all $D_\pi=0$ we are left with an action that depends 
%% directly on a combination of the Riemann and Torsion tensor. 
%% Determination of the equations of motion is straightforward, and we obtain 
%% a combination of $\nabla_\mu T_{\nu}{}^\sigma{}_{\lambda}$ and 
%% $R_{\mu\nu}{}^\sigma{}_{\lambda}$. However, due to the complicated 
%% combination of invariant tensors, we can solve these equations by setting
%%  $\nabla_\mu T_{\nu}{}^\sigma{}_{\lambda}=0$ and 
%% $R_{\mu\nu}{}^\sigma{}_{\lambda}=0$. 

%% Thus, it seems appropriate to decompose it in terms of $
%% \Gamma_{\nu}{}^\lambda{}_{\rho}=\hat{\Gamma}_{\nu}{}^\lambda{}_{\rho}+\hat{ T}_{\nu}{}^\lambda{}_{\rho}+A_{[\nu}\delta^\lambda_{\rho]}$. Then,
%% \ba
%% R_{\mu\nu}{}^\lambda{}_\rho&=&2\hat{\nabla}_{[\mu}\hat{\Gamma}_{\nu]}{}^\lambda{}_{\rho}+2\hat{\nabla}_{[\mu}\hat{ T}_{\nu]}{}^\lambda{}_{\rho}-2\hat{ T}_{\sigma}{}^\lambda{}_{[\mu}\hat{ T}_{\nu]}{}^\sigma{}_{\rho}+\\ \nonumber&&+\hat{\nabla}_{[\mu}A_{\nu]}\delta^\lambda_\rho+\delta^\lambda_{[\mu}\hat{\nabla}_{\nu]}A_\rho-\hat{ T}_\mu{}^\lambda{}_{\nu}A_\rho-\delta^\lambda_{[\mu}\hat{ T}_{\nu]}{}^\sigma{}_{\rho}A_\sigma +\frac{1}{2}\delta^\lambda_{[\mu}A_{\nu]}A_\rho
%% \ea

%% {\bf esto se ve tan feo que creo que es preferible trabajar con la torsion sin hacer la descomposicion de la torsion en $\hat{ T} +A$}

%% Thus, we decompose the connection in terms of its symmetric and antisymmetric parts $
%% \Gamma_{\nu}{}^\lambda{}_{\rho}=\hat{\Gamma}_{\nu}{}^\lambda{}_{\rho}+\frac{1}{2} T_{\nu}{}^\lambda{}_{\rho}$
%% \be
%% R_{\mu\nu}{}^\lambda{}_\
