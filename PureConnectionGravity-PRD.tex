\documentclass[%twocolumn,
  showpacs,showkeys,prd,superscriptaddress]{revtex4-1}

\usepackage{amsmath,amsthm,latexsym,amssymb,amsfonts,epsfig}
\usepackage{xcolor}
\usepackage{ulem}
%\usepackage{authblk}
\usepackage[%
  colorlinks=true,
  urlcolor=blue,
  linkcolor=blue,
  citecolor=blue
]{hyperref}

%------------------
%--------- Definitions
%------------------
%% \def\be{\begin{equation}}
%% \def\ee{\end{equation}}
%% \def\ba{\begin{eqnarray}}
%% \def\ea{\end{eqnarray}}
\def\md{{\mathrm{d}}}
%\def\mx{{\mathrm{x}}}
\def\my{{\mathrm{y}}}
\def\mz{{\mathrm{z}}}
\def\l{\left}
\def\r{\right}
\def\ll{\left\{}
\def\rl{\right\}}
\def\lp{\left(}
\def\rp{\right)}
\def\lc{\left[}
\def\rc{\right]}
%% \renewcommand{\{}{\left\{}
%% \renewcommand{\}}{\right\}}
%% \renewcommand{\[}{\left[}
%% \renewcommand{\]}{\right]}
%% \renewcommand{\(}{\left(}
%% \renewcommand{\)}{\right)}
\def\la{\left\langle}
\def\ra{\right\rangle} 
%% \DeclareMathOperator{\det}{det}
%% \newcommand\tor[3]{\ensuremath{T_{#1}{}^{#2}{}_{#3}}}
%% %\renewcommand\Authands{ and }
%% \newcommand{\out}[1]{{\color{red} \sout{#1}}}
%% \newcommand{\pro}[1]{{\color{blue} #1}}

\usepackage{amsmath,amssymb,amsfonts,dsfont,mathrsfs,amsthm}
\usepackage{graphicx}
\usepackage{centernot}
\usepackage{hyperref}
\usepackage{xcolor}
%% \usepackage{comment}
\hypersetup{linktocpage,colorlinks=true,urlcolor=blue!80!red,linkcolor=blue,citecolor=red}
%% \usepackage{feynmf}
\usepackage{siunitx}
\usepackage{array}
\usepackage{ulem}
%% \usepackage{tikz}
\usepackage{braket}
%% \usetikzlibrary{shapes,
%%   snakes,
%%   decorations.pathmorphing,
%%   decorations.markings,
%%   calc,
%%   shadows.blur,
%%   shadings}
%% \usepackage[framemethod=tikz]{mdframed}

%% \makeatletter
%% % gluon decoration (based on the original coil decoration)
%% \pgfdeclaredecoration{gluon}{coil}
%% {
%%   \state{coil}[switch if less than=%
%%     0.5\pgfdecorationsegmentlength+%>
%%     \pgfdecorationsegmentaspect\pgfdecorationsegmentamplitude+%
%%     \pgfdecorationsegmentaspect\pgfdecorationsegmentamplitude to last,
%%                width=+\pgfdecorationsegmentlength]
%%   {
%%     \pgfpathcurveto
%%     {\pgfpoint@oncoil{0    }{ 0.555}{1}}
%%     {\pgfpoint@oncoil{0.445}{ 1    }{2}}
%%     {\pgfpoint@oncoil{1    }{ 1    }{3}}
%%     \pgfpathcurveto
%%     {\pgfpoint@oncoil{1.555}{ 1    }{4}}
%%     {\pgfpoint@oncoil{2    }{ 0.555}{5}}
%%     {\pgfpoint@oncoil{2    }{ 0    }{6}}
%%     \pgfpathcurveto
%%     {\pgfpoint@oncoil{2    }{-0.555}{7}}
%%     {\pgfpoint@oncoil{1.555}{-1    }{8}}
%%     {\pgfpoint@oncoil{1    }{-1    }{9}}
%%     \pgfpathcurveto
%%     {\pgfpoint@oncoil{0.445}{-1    }{10}}
%%     {\pgfpoint@oncoil{0    }{-0.555}{11}}
%%     {\pgfpoint@oncoil{0    }{ 0    }{12}}
%%   }
%%   \state{last}[next state=final]
%%   {
%%     \pgfpathcurveto
%%     {\pgfpoint@oncoil{0    }{ 0.555}{1}}
%%     {\pgfpoint@oncoil{0.445}{ 1    }{2}}
%%     {\pgfpoint@oncoil{1    }{ 1    }{3}}
%%     \pgfpathcurveto
%%     {\pgfpoint@oncoil{1.555}{ 1    }{4}}
%%     {\pgfpoint@oncoil{2    }{ 0.555}{5}}
%%     {\pgfpoint@oncoil{2    }{ 0    }{6}}
%%   }
%%   \state{final}{}
%% }

%% \def\pgfpoint@oncoil#1#2#3{%
%%   \pgf@x=#1\pgfdecorationsegmentamplitude%
%%   \pgf@x=\pgfdecorationsegmentaspect\pgf@x%
%%   \pgf@y=#2\pgfdecorationsegmentamplitude%
%%   \pgf@xa=0.083333333333\pgfdecorationsegmentlength%
%%   \advance\pgf@x by#3\pgf@xa%
%% }
%% \makeatother

%% \tikzset{
%%   boson/.style={decorate,decoration={gluon,segment length=9pt,aspect=0}},
%%   % style to apply some styles to each segment of a path
%%   on each segment/.style={
%%     decorate,
%%     decoration={
%%       show path construction,
%%       moveto code={},
%%       lineto code={
%%         \path [#1]
%%         (\tikzinputsegmentfirst) -- (\tikzinputsegmentlast);
%%       },
%%       curveto code={
%%         \path [#1] (\tikzinputsegmentfirst)
%%         .. controls
%%         (\tikzinputsegmentsupporta) and (\tikzinputsegmentsupportb)
%%         ..
%%         (\tikzinputsegmentlast);
%%       },
%%       closepath code={
%%         \path [#1]
%%         (\tikzinputsegmentfirst) -- (\tikzinputsegmentlast);
%%       },
%%     },
%%   },
%%   % style to add an arrow in the middle of a path
%%   mid arrow/.style={postaction={decorate,decoration={
%%         markings,
%%         mark=at position .5 with {\arrow[#1]{stealth}}
%%       }}},
%% }


%-------------------------------Theorems
\newtheorem{Def}{Definition}
\newtheorem{Thm}{Theorem}
\newtheorem{Lem}{Lemma}
\newtheorem{Pos}{Postulate}
\newtheorem{Exa}{Example}
\newtheorem{Cor}{Corrolary}
\newtheorem{Pro}{Proposition}

%---------------------------------New commands
\newcommand{\A}{\mathcal{A}} 
\newcommand{\abs}[1]{\left|{#1}\right|}
\newcommand{\Ag}{\mathcal{A}_{(1)}}
\newcommand{\Agf}{\boldsymbol{\mathcal{A}}}
\newcommand{\Af}{\, {\mathbf{A}} }
\newcommand{\AF}[1]{\, {\mathbf{A}}_{({#1})} }
\newcommand{\hAf}{\, \hat{\mathbf{A}}_{(1)} }
\newcommand{\hAF}[1]{\, \hat{\mathbf{A}}_{(#1)} }
\newcommand{\bboxed}[1]{{\color{red}{\boxed{\boxed{\textcolor{black}{#1}}}}}}
\newcommand{\C}{\mathbb{C}}
\newcommand{\Cl}{\mathcal{C}\!\ell}
\newcommand{\cdf}[1][]{\,{\boldsymbol{\mathcal{D}}}{#1}\!}
\newcommand{\covd}{\mathcal{D}}
\newcommand{\D}{\mathscr{D}}
\newcommand{\df}[1][]{\,{\mathbf{d}}{#1}\!}
\newcommand{\dfd}{\,{\mathbf{d}}^\dag\!}
\newcommand{\ele}[2][]{\frac{d}{dt}\left(\frac{\partial\mathcal{L}}{\partial \dot{#2}^{#1}}\right) - \frac{\partial\mathcal{L}}{\partial {#2}^{#1}}}
\newcommand{\fele}[2][]{\partial_\mu \left(\frac{\delta\mathcal{L}}{\delta\left(\partial_\mu{#2}^{#1}\right)}\right) - \frac{\delta\mathcal{L}}{\delta {#2}^{#1}}}
\newcommand{\vb}[1]{\vec{e}_{#1}}
%% \newcommand{\fb}[1]{\widetilde{e}{}^{\, #1}}
\newcommand{\fb}[1]{\widetilde{e}{}^{\, #1}}
\newcommand\fder[3][]{\frac{\delta^{#1}{#2}}{\delta {#3}^{#1}}}
\newcommand\fdern[4][]{\frac{\delta^{#1}{#2}}{\delta {#3} \cdots \delta {#4}}}
\newcommand{\F}{\,\boldsymbol{\mathcal{F}}}
\newcommand{\Fg}{\,\boldsymbol{\mathcal{F}}_{(2)}}
\newcommand{\Ff}{\,{\mathbf{F}}}
\newcommand{\FF}[1]{\,{\mathbf{F}}_{(#1)}}
\newcommand{\hFF}[1]{\,{\mathbf{\hat{F}}}_{(#1)}}
\newcommand{\fy}{\centernot}
\newcommand{\G}{\mathscr{G}}
\newcommand{\ga}{\gamma}
\newcommand{\gf}{\,\boldsymbol{\gamma}}
\newcommand{\Ga}{\Gamma}
\newcommand{\conn}[3]{\left(\Gamma_{#1}\right)^{#2}{}_{#3}}
\newcommand{\Ha}{\mathscr{H}}
\newcommand{\He}{\mathbb{H}}
\newcommand{\Hi}{\mathcal{H}}
\newcommand{\Hint}{\underline{\sc Hint:} }
\DeclareMathOperator{\hs}{\,\star\!\!}
\newcommand{\J}{\mathscr{J}}
\newcommand{\K}{\mathbb{K}}
\newcommand{\KK}{Kaluza-Klein{\;}}
\newcommand{\Lag}{\mathscr{L}}
\newcommand{\Li}{\mathcal{L}}
\newcommand{\La}[1][]{\triangle_{#1}}
\newcommand{\Lap}{\nabla^2}
\newcommand{\lr}[1]{\stackrel{\leftrightarrow}{#1}}
\newcommand{\M}{\ensuremath{\mathscr{M}}}
\newcommand{\Mi}{\mathcal{M}}
\newcommand{\MN}{Maldacena-N\'u\~nez{\;}}
\newcommand{\N}{\ensuremath{\mathscr{N}}}
\newcommand{\Na}{\mathbb{N}}
\newcommand{\No}{\mathcal{N}}
\newcommand{\norm}[1]{\left\|#1\right\|}
\newcommand{\Op}{\mathcal{O}}
\newcommand{\Or}{\mathscr{O}}
\newcommand\pder[3][]{\frac{\partial^{#1}{#2}}{\partial {#3}^{#1}}}
\newcommand\pdern[4][]{\frac{\partial^{#1}#2}{\partial #3\cdots\partial #4}}
\newcommand{\Qh}[1][]{\ensuremath{\hat{Q}_{#1}}}
\newcommand{\R}{\mathbb{R}}
\newcommand{\Ri}{\mathcal{R}}
%% \newcommand{\S}{\mathscr{S}}
\newcommand{\SM}{Standard Model {}}%\mathscr{S}}
\newcommand{\T}{\mathscr{T}}
\newcommand{\tor}{\mathcal{T}}
\newcommand{\tors}[3]{\mathcal{T}{}_{#1}{}^{#2}{}_{#3}}
\newcommand\vdj[1]{\left< \Delta J \right>_{#1}}
\newcommand{\w}{{\scriptstyle\wedge}\!}
\newcommand{\Z}{\mathbb{Z}}

\newcommand{\dbar}[1]{\ensuremath{\mathchar'26\mkern-12mu \mathrm{d}^{#1}}\!}


%--------------------------- New Greek
\newcommand{\tht}{\ensuremath{\theta}}
\newcommand{\bet}{\ensuremath{\bar{\eta}}}
\newcommand{\bps}{\ensuremath{\bar{\psi}}}
\newcommand{\bc}{\ensuremath{\bar{\chi}}}
\newcommand{\Bps}{\ensuremath{\bar{\Psi}}}
\newcommand{\Bx}{\ensuremath{\bar{\Xi}}}
\newcommand{\bph}{\ensuremath{\bar{\phi}}}
\newcommand{\vph}{\ensuremath{\varphi}}
\newcommand{\bvph}{\ensuremath{\bar{\varphi}}}
\newcommand{\bth}{\ensuremath{\bar{\theta}}}
\newcommand{\hph}{\ensuremath{\hat{\phi}}}

\newcommand{\bs}[1]{\boldsymbol{#1}}


\renewcommand{\div}{{\mathbf{div}}}
\newcommand{\grad}{{\mathbf{grad}}}
\newcommand{\curl}{{\mathbf{curl}}}

\newcommand\VI[2]{\,\hat{e}^{\hat{#1}}_{\hat{#2}}}
\newcommand\VIF[1]{\,\hat{\boldsymbol{e}}^{\hat{#1}}}
\newcommand\VIN[2]{\;\hat{E}^{\hat{#1}}_{\hat{#2}}}
\newcommand\VINF[1]{\,\hat{\mathbf{E}}_{\hat{#1}}}
\newcommand\hvi[2]{\;\hat{e}^{{#1}}_{{#2}}}
\newcommand\hvin[2]{\;\hat{E}^{{#1}}_{{#2}}}
\newcommand\hvif[1]{\,\hat{\mathbf{e}}^{{#1}}}
\newcommand\hvinf[1]{\,\hat{\mathbf{E}}_{{#1}}}
\newcommand\vi[2]{e^{{#1}}_{{#2}}}
\newcommand\vin[2]{E^{{#1}}_{{#2}}}
\newcommand\vif[1]{\,{\mathbf{e}}^{{#1}}}
\newcommand\vinf[1]{\,{\mathbf{E}}_{{#1}}}
%% \newcommand\Vi[2]{e^{\hat{#1}}_{\hat{#2}}}
%% \newcommand\Vin[2]{E^{\hat{#1}}_{\hat{#2}}}
\newcommand\GAM[1]{{\gamma}^{\hat{#1}}}
%% \newcommand\Gam[1]{\gamma^{\hat{#1}}} 
\newcommand\gam[1]{\gamma^{{#1}}}
\newcommand\hgam[1]{\hat{\gamma}^{{#1}}}
\newcommand\NAB[1]{\hat{\nabla}_{\hat{#1}}}
\newcommand\Nab[1]{\nabla_{\hat{#1}}}
\newcommand\nab[1]{\nabla_{{#1}}}
\newcommand\PA[1]{\partial_{\hat{#1}}}
\newcommand\pa[1]{\partial_{{#1}}}
\newcommand\PAU[1]{\partial^{\hat{#1}}}
\newcommand\pau[1]{\partial^{{#1}}}
\newcommand\lf[1]{{\omega}^{{#1}}}
\newcommand\lft[1]{\hat{\omega}^{{#1}}}
\newcommand\SPI[1]{\;\hat{\omega}_{\hat{#1}}}
\newcommand\SPIF[2]{\,\hat{\boldsymbol{\omega}}^{\hat{#1}}{}_{\hat{#2}}}
%% \newcommand\Spi[1]{\omega_{\hat{#1}}}
\newcommand\spi[1]{\omega_{{#1}}}
\newcommand\tspi[1]{\tilde{\omega}_{{#1}}}
\newcommand\spif[2]{\,{\boldsymbol{\omega}}^{{#1}}{}_{{#2}}}
\newcommand\tspif[2]{\,{\tilde{\boldsymbol{\omega}}}^{{#1}}{}_{{#2}}}
\newcommand\hspi[1]{\hat{\omega}_{{#1}}}
\newcommand\hspif[2]{\,\hat{\boldsymbol{\omega}}^{{#1}}{}_{{#2}}}
%%%%%%%%% Beware of the inconsistency between
%%%%%%%%% \Rif and \RIF
\newcommand{\RIF}[2]{\,\hat{\boldsymbol{\mathcal{R}}}^{\hat{#1}}{}_{\hat{#2}}}
\newcommand{\hRif}[2]{\,\hat{\boldsymbol{\mathcal{R}}}^{{#1}}{}_{{#2}}}
\newcommand{\Rif}[2]{\,\boldsymbol{\mathcal{R}}^{{#1}}{}_{{#2}}}
\newcommand{\tRif}[2]{\,\tilde{\boldsymbol{\mathcal{R}}}^{{#1}}{}_{{#2}}}
\newcommand{\Tf}[1]{\,\boldsymbol{\mathcal{T}}^{#1}}
\newcommand{\TF}[1]{\,\hat{\boldsymbol{\mathcal{T}}}^{\hat{#1}}}
\newcommand{\Tor}[2]{\mathcal{T}^{#1}{}_{#2}}
\newcommand{\cont}[3]{\mathcal{K}_{#1}{}^{#2}{}_{#3}}
\newcommand{\contf}[2]{\,\boldsymbol{\mathcal{K}}^{#1}{}_{#2}}
\newcommand{\hcont}[3]{\hat{\mathcal{K}}_{#1}{}^{#2}{}_{#3}}
\newcommand{\hcontf}[2]{\,\hat{\boldsymbol{\mathcal{K}}}^{#1}{}_{#2}}
\newcommand{\CONT}[3]{\hat{\mathcal{K}}_{\hat{#1}}{}^{\hat{#2}}{}_{\hat{#3}}}
\newcommand{\CONTF}[2]{\,\hat{\boldsymbol{\mathcal{K}}}^{\hat{#1}}{}_{\hat{#2}}}

%% \newcommand{\ket}[1]{\left.\left|#1\right.\right>}
%% \newcommand{\bra}[1]{\left.\left<#1\right.\right|}
\renewcommand\bra[1]{\Bra{#1}}
\renewcommand\ket[1]{\Ket{#1}}
\newcommand{\bkt}[3]{\Braket{ {#1} | {#2} | {#3} } }
\newcommand{\bk}[2]{\Braket{ {#1} | {#2} } }
\newcommand{\comm}[2]{\left[#1,#2\right]}
\newcommand{\acomm}[2]{\left\{#1,#2\right\}}
\newcommand{\vev}[1]{\ensuremath{\left<#1\right>}}
\renewcommand{\set}[1]{\ensuremath{\Set{ #1 }}}

\newcommand{\relphantom}[1]{\mathrel{\phantom{#1}}}

\newcommand{\Ric}{\operatorname{Ric}}
\newcommand*{\diag}{\operatorname{diag}}
\newcommand{\id}{\operatorname{id}}
\newcommand{\tr}{\operatorname{tr}}
\newcommand{\Tr}{\operatorname{Tr}}
\newcommand{\Ker}{\operatorname{Ker}}
\renewcommand{\Im}{\operatorname{Im}}
\newcommand{\sgn}{\operatorname{sgn}}
\newcommand{\Ln}{\operatorname{Ln}}
\newcommand{\Ei}{\operatorname{Ei}}
\newcommand{\csch}{\operatorname{csch}}
\newcommand{\arcsinh}{\operatorname{arcsinh}}
\DeclareMathOperator\Br{Br}

\newcommand{\beq}{\begin{equation}}
\newcommand{\eeq}{\end{equation}}
\newcommand{\ber}{\begin{eqnarray}}
\newcommand{\eer}{\end{eqnarray}}

\renewcommand{\(}{\left(}
\renewcommand{\)}{\right)}
\renewcommand{\[}{\left[}
\renewcommand{\]}{\right]}

\newcommand{\uf}[2][]{\ensuremath{u_{#1}\(\vec{#2}\)}}
\newcommand{\ufb}[2][]{\ensuremath{\bar{u}_{#1}\(\vec{#2}\)}}
\newcommand{\vf}[2][]{\ensuremath{v_{#1}\(\vec{#2}\)}}
\newcommand{\vfb}[2][]{\ensuremath{\bar{v}_{#1}\(\vec{#2}\)}}
\newcommand\pol[2][]{\ensuremath{\varepsilon_{#1}(\vec{#2})}}
\newcommand\polc[2][]{\ensuremath{\varepsilon^*_{#1}(\vec{#2})}}
\newcommand{\ann}[3]{\ensuremath{#1\(\vec{#2},#3\)}}
\newcommand{\cre}[3]{\ensuremath{#1^\dag\(\vec{#2},#3\)}}
%% \newcommand{\uf}[2]{\ensuremath{u\(\vec{#1},#2\)}}
%% \newcommand{\ufb}[2]{\ensuremath{\bar{u}\(\vec{#1},#2\)}}
%% \newcommand{\vf}[2]{\ensuremath{v\(\vec{#1},#2\)}}
%% \newcommand{\vfb}[2]{\ensuremath{\bar{v}\(\vec{#1},#2\)}}
%% \newcommand{\ann}[3]{\ensuremath{#1\(\vec{#2},#3\)}}
%% \newcommand{\cre}[3]{\ensuremath{#1^\dag\(\vec{#2},#3\)}}

\newcommand{\dif}{{\mathrm{d}}}
\newcommand{\difn}[1]{{\mathrm{d}}^{#1}}
\newcommand{\dn}[2]{\,{\mathrm{d}}^{#1}\!{#2}\;}
\newcommand{\dbarn}[2]{\,\dbar{#1}{#2}\;}
\newcommand*{\de}[1]{\mathop{\mathrm{d}#1}\nolimits}% differential, facultative argoment between square parentheses
\newcommand*{\desec}[1][]{\mathop{\mathrm{d^2}#1}\nolimits}% second differential, facultative argoment between square parentheses
\newcommand{\der}[2]{\frac{\de{#1}}{\de{#2}}}% first derivative 
%\newcommand{\pder}[2]{\frac{\pa{}#1}{\pa{}{#2}}}% first derivative 
\newcommand{\inlineder}[2]{\mathrm{d}{#1}/\mathrm{d}{#2}}% in-line first derivative
\newcommand{\dersec}[2]{\frac{{\desec[#1]}}{\de[{#2}^2]}}% second derivative
%% \newcommand{\dx}{\de[x]}% frequently used differentials
%% \newcommand{\dy}{\de[y]}
%\newcommand{\df}{\de[f]}

%------------------
%--------- Format
%------------------
\newcommand{\out}[1]{{\color{red} \sout{#1}}}
\newcommand{\pro}[1]{{\color{blue!70!black} #1}}




\hypersetup{%
  pdftitle={Minimal Affine Gravity},
  pdfauthor={Aureliano Skirzewski,}{Oscar Castillo-Felisola},
  pdfkeywords={Affine Gravity,} {Torsion,} {Generalised Gravity.},
  pdflang={English}
}

%% %-------------------
%% %--------- Title Page
%% %------------------
%% \title{Minimal Affine Gravity.}

%% \author[1]{{\sf Aureliano Skirzewski}\thanks{\texttt{askirz@gmail.com, skirz@ula.ve}}}
%% \author[2,3]{{\sf Oscar Castillo-Felisola}\thanks{\texttt{o.castillo.felisola@gmail.com}}}

%% \affil[1]{\small Centro de F\'isica Fundamental,  Universidad de los Andes, 5101 M\'erida, Venezuela.} 
%% \affil[2]{\small Departamento de F\'isica, Universidad Santa Mar\'ia, Valpara\'iso--Chile}
%% \affil[3]{\small Centro Cient\'ifico y Tecnol\'ogico de Valpara\'iso, Chile.}

%% \date{}

%------------------
%--------- Document
%------------------
\begin{document}

\title{Minimal Affine Gravity}


\author{Aureliano \surname{Skirzewski}}
\email[Corresponding Author: ]{skirz@ula.ve}
\affiliation{Centro de F\'isica Fundamental,  Universidad de los Andes, 5101 M\'erida, Venezuela.}

\author{Oscar \surname{Castillo-Felisola}}
\email{o.castillo.felisola@gmail.com}
\affiliation{Centro Cient\'\i fico Tecnol\'ogico de Valpara\'\i so, Chile.}
\affiliation{Departamento de F\'\i sica, Universidad T\'ecnica Federico Santa Mar\'\i a, Casilla 110-V, Valpara\'\i so, Chile.}

%--------- Abstract
\begin{abstract}
We develop a topological theory of gravity with torsion where metric arises as a constructed field. This approach towards gravity resembles pre-geometrical approaches in that a fundamental metric is not assumed even though the Torsion gives place to a spacetime structure and a Newtonian like gravitational force is obtained in the perturbative limit of the theory.
\end{abstract}

\pacs{04.50.-h,04.62.+v,11.25.Mj,11.25.-w}
\keywords{Affine Gravity, Torsion, Generalised Gravity.}

\maketitle

\section{Introduction}

General Relativity (GR) is nowadays the accepted theory of gravitation. The gravitational effects are interpreted as a geometrical influence due to the curvature of the spacetime. In principle, the spacetime is a (pseudo-) Riemannian manifold whose metric tensor is the fundamental field in the description of the gravity.

More generally, the spacetime might be associated with a topological space, $\mathcal{M}$, with less restrictive structure than a (pseudo-) Riemannian geometry  by choosing a connection non-compatible with the metric, in the sense that $\nabla g\neq 0$. It is  still possible to define a connection associated with the parallel transport of vectors along curves on the spacetime manifold. Interestingly, this general connection cannot be written in terms of the metric, but decomposes into the usual Levi-Civita connection plus a contribution from the torsion tensor,
\begin{align}
  \Gamma^{\mu}{}_{\rho\sigma} = \hat{\Gamma}^{\mu}{}_{\rho\sigma} + T^{\mu}{}_{\rho\sigma},
\end{align}
where the hatted objects represent the the torsion-free part (in the case of the connection it is just the Levi-Civita connection).

Therefore, another possible formulation of a gravitational theory within the context of a general connection would have an action like,
\begin{align}
  S[g,\Ga]=\int\dn{n}{x}\sqrt{g}\; g^{\mu\nu}\Ri_{\mu\nu}\(\Ga,\partial\Ga\).
\end{align}
This theory is known as the metric-affine gravity. It should be noticed that the above generalization of gravity is ``minimal'' in the sense that the new tensor (the torsion) enters only through the minimal coupling, while more general actions admit quadratic terms in the torsion.

\vspace*{5mm}

\hrule

\vspace*{5mm}

The aim of this work is to present a ``minimal'' affine theory of gravity, where the symmetric tensor which acts like ``metric'' is a composite field defined in terms of the connection,
\begin{align}
  g_{\mu\nu} = T^\sigma{}_{\rho\mu} T^\rho_{\sigma\nu}.
\end{align}

The structure of the paper ...

%A gravitational theory can still be formulated in this non-metric spacetimes, and are known as \textit{affine gravity}.



%--------- Scattered Ideas
\section{Three-dimensional case}

We will start with some definitions.
Formally, torsion is defined through 
%\be \label{def1}\l[\nabla _\mu,\nabla_\nu\r]f(x) =T_\mu{}^\rho{}_\nu\partial_\rho f(x),\ee similarly
\begin{align}
  \label{curvdef}
  \lc \nabla_\mu,\nabla_\nu\rc \omega_\lambda=
  -T_\mu{}^\rho{}_\nu\nabla_\rho\omega_\lambda + R_{\mu\nu}{}^\rho{}_\lambda \omega _\rho,
\end{align}
where $\nabla_\nu$ is expressed through a connection 
$\Gamma_\mu{}^\nu{}_\lambda$ and we can identify its antisymmetric part with torsion $T_\mu{}^\nu{}_\lambda=2\Gamma_{[\mu}{}^\nu{}_{\lambda]}.$
It can be checked easily that the torsion tensor $T_\mu{}^\lambda{}_\nu$ 
is a covariant tensor under diffeomorphisms. Thus, in order to build
topological invariants of density one, we can use the skew symmetric 
Levi-\v{C}ivita tensor $\epsilon^{\mu_1\mu_2\dots\mu_n}$ in $n$-dimensional space(-time).
With these ingredients, in 3 dimensional space we can write the action 
\begin{align}
  \label{accion3d}
  S[\Gamma] &=
  \int\md^3x \Bigg\{R_{\mu_1\mu_2}{}^\rho{}_{\mu_3} T_{\mu_4}{}^\sigma{}_{\mu_5}\sum_{\pi \in \mathrm Z_5}C_\pi\delta_\rho^{\mu_{\pi(1)}} \delta_\sigma^{\mu_{\pi(2)}}\epsilon^{\mu_{\pi(3)}\mu_{\pi(4)}\mu_{\pi(5)}} \notag\\ 
  & \qquad + T_{\mu_1}{}^\rho{}_{\mu_2} T_{\mu_3}{}^\sigma{}_{\mu_4}T_{\mu_5}{}^\tau{}_{\mu_6} \sum_{\pi \in \mathrm Z_6}D_\pi\delta_\rho^{\mu_{\pi(1)}} \delta_\sigma^{\mu_{\pi(2)}}\delta_\tau^{\mu_{\pi(3)}}\epsilon^{\mu_{\pi(4)}\mu_{\pi(5)}\mu_{\pi(6)}}\Bigg\}, 
\end{align}
where we have included all possible permutations of $n$ elements $\pi\in\mathrm Z_n$ in the sums with a different constant $C_\pi$ and $D_\pi$ for each of the permutations. 
In particular, $\pi(i)\in\mathrm Z_n$ if $i\in\mathrm Z_n$ represents a permutation of the integer numbers between $1$ and $n$.

After an invariant tensor decomposition of the torsion field 
\begin{align}
  T_{\mu}{}^\sigma{}_{\nu}=\epsilon_{\mu\nu\rho} T^{\sigma\rho}+A_{[\mu}\delta^\sigma{}_{\nu]}
\end{align}
with a symmetric $T^{\sigma\rho}$ of density weight  $w =1$. Notice that $T_{\mu}{}^\nu{}_{\nu}=A_{\mu}$ is the trace part of the more arbitrary $T_{\mu}{}^\sigma{}_{\nu}$.

The action \eqref{accion3d} can be rewritten as 
\begin{align}
  \label{accion3dnew}
  S[\Gamma]&=\int\md^3x \Big\{ B_1R_{\mu\nu}{}^\mu{}_{\rho} T^{\nu\rho} + B_2\epsilon^{\mu\nu\rho}R_{\mu\nu}{}^{\sigma}_\sigma A_\rho + B_3\epsilon^{\nu\rho\sigma}R_{\mu\nu}{}^{\mu}{}_\rho A_\sigma\notag\\
  & \qquad + B_4 \det(T^{\mu\nu}) + B_5 \epsilon^{\mu\nu\rho}R_{\mu\nu}{}^\lambda{}_\rho A_\lambda + B_6 T^{\mu\nu}A_\mu A_\nu \Big\}
\end{align}
with the connection $\Gamma_{\mu}{}^\lambda{}_\rho=\hat\Gamma_{\mu}{}^\lambda{}_\rho+\epsilon_{\mu\rho\sigma}T^{\lambda\sigma}+A_{[\mu}\delta^\lambda{}_{\rho]}$ decomposed into its symmetric and antisymmetric parts
\begin{align}
  \label{RiemmanDecomposition}
  R_{\mu\nu}{}^\lambda{}_\rho&=
  \hat{R}_{\mu\nu}{}^\lambda{}_\rho+\epsilon_{\kappa\rho[\mu}\hat\nabla_{\nu]}T^{\kappa\lambda}+\frac{1}{2}\partial_{[\mu}A_{\nu]}\delta^\lambda_\rho+\frac{1}{2}\delta^\lambda_{[\mu}\hat\nabla_{\nu]}A_\rho-\frac{1}{4}\epsilon_{\mu\nu\kappa}T^{\kappa\lambda}A_\rho\\\nonumber
    & \quad -\frac{1}{4}\delta^\lambda_{[\mu}\epsilon_{\nu]\rho\tau}T^{\tau\sigma}A_\sigma + \frac{1}{8}\delta_{[\mu}^\lambda A_{\nu]}A_\rho-\frac{1}{2}\epsilon_{\kappa\sigma[\mu}\epsilon_{\nu]\rho\tau}T^{\lambda\kappa}T^{\sigma\tau},
\end{align}
where $ \hat\nabla_\rho$ and $\hat{R}_{\mu\nu}{}^\lambda{}_\rho$ are the covariant derivative and the curvature associated to the symmetric part of the connection.
Notice that Bianchi identity, obtained as $\epsilon^{\mu\nu\lambda}\hat R_{\mu\nu}{}^\rho{}_\lambda=0$, leads us to the following
\begin{align}
  \label{bianchi}
  \epsilon^{\mu\nu\rho} R_{\mu\nu}{}^\lambda{}_\rho = \hat\nabla_\rho T^{\rho\lambda}+\epsilon^{\mu\nu\lambda}\partial_\mu A_\nu -\frac{1}{2}T^{\lambda\rho}A_\rho. 
\end{align}
Using curvature's decomposition \eqref{RiemmanDecomposition}  we can rewrite the action \eqref{accion3dnew} as
\begin{align}
  \label{accion3dnewer}
  S[\Gamma]
  &= \int\md^3x \Bigg\{B_1\hat R_{\mu\nu}{}^\mu{}_{\rho} T^{\nu\rho}+B_2\epsilon^{\mu\nu\rho}\hat R_{\mu\nu}{}^{\sigma}{}_\sigma A_\rho+B_3\epsilon^{\nu\rho\sigma}\hat R_{\mu\nu}{}^{\mu}{}_\rho A_\sigma \nonumber\\ 
  & \qquad +\l(B_4-\frac{3}{4}B_1\r) \det(T^{\mu\nu}) + \l(B_3-\frac{1}{2}B_1\r) A_\nu\hat\nabla_\mu T^{\mu\nu}\nonumber\\ 
  & \qquad + B_2\epsilon^{\mu\nu\rho}A_\mu\partial_\nu A_\rho+\frac{1}{4}(-2B_2+B_3)T^{\mu\nu}A_\mu A_\nu\Bigg\}
\end{align}
substitution of \eqref{bianchi} in \eqref{accion3dnewer} and direct evaluation will lead us to
\begin{align}
  \label{accion3dfinal}
  S[\Gamma]&=\int\md^3x \Bigg\{B_1\hat R_{\mu\nu}{}^\mu{}_{\rho} T^{\nu\rho}+(B_2-\frac{1}{2}B_3)\epsilon^{\mu\nu\rho}A_\mu\hat R_{\nu\rho}{}^{\sigma}{}_\sigma \nonumber\\
  & \qquad + \frac{1}{4}(4B_4-3B_1) \det(T^{\mu\nu})\\
  & \qquad + \frac{1}{4}(B_1+4B_2-2B_3)\epsilon^{\mu\nu\rho}A_\mu\partial_\nu A_\rho+\frac{1}{4}(-2B_2+B_3)T^{\mu\nu}A_\mu A_\nu\Bigg\}.\notag
\end{align}

An interesting sector of the theory corresponds to the space of non-degenerated $T^{\mu\nu}$ because rewritting it as $\sqrt{g}g^{\mu\nu},$ reveals a one to one correspondence with General Relativity non minimally coupled to the $A_\mu$ field
\begin{align}
  \label{accion3dGR}
  S[g,\hat\Gamma,A]&=
  \int\md^3x \l\{\sqrt{g}\l(B_1\hat R +\frac{1}{4}(4B_4-3B_1) -\frac{1}{4}\l( 2B_2-B_3\r) A_\mu A^\mu\r)\r.\\
  & \qquad \l.+\frac{1}{2}(2B_2-B_3)\epsilon^{\mu\nu\rho}A_\mu\hat R_{\nu\rho}{}^{\sigma}{}_\sigma+\frac{1}{4}(B_1+4B_2-2B_3)\epsilon^{\mu\nu\rho}A_\mu\partial_\nu A_\rho\r\}\notag
\end{align}
as a matter of fact, the vector field and the ``Gravitational" sector of the Torsion disengage when $B_3=2B_2$

After these considerations we can conclude that the final expression for the action functional is then supposed to be expressed as a linear combination of all the invariants we can construct from $\hat\Gamma_{\mu}{}^{\nu}{}_\rho$, $T^{\mu\nu}$ and $A_\mu$. Thus, the most general action should be 
\begin{equation}
  \begin{split}
    S[\hat\Gamma_{\mu}{}^{\nu}{}_\rho,T^{\mu\nu},A_\mu] = \int \md^3x & \bigg(
    C_1\hat R_{\mu\nu}{}^{\mu}{}_\rho T^{\nu\rho}\\
    & + \epsilon^{\mu\nu\rho}\big(C_2\hat R_{\mu\nu}{}^{\sigma}{}_\sigma A_\rho
    + C_3A_\mu\partial_\nu A_\rho\big)\\%\nonumber
    & + C_4T^{\mu\nu}\hat\nabla_\mu A_\nu
    + C_5T^{\mu\nu}A_\mu A_\nu
    + C_6\det(T)\\%\nonumber
    & + C_7\epsilon^{\mu\nu\lambda}\big(\hat\Gamma_{\mu}{}^{\sigma}{}_\rho\partial_\nu\hat\Gamma_{\lambda}{}^{\rho}{}_\sigma+\frac{2}{3}\hat\Gamma_{\mu}{}^{\tau}{}_\rho\hat\Gamma_{\nu}{}^{\rho}{}_\sigma\hat\Gamma_{\lambda}{}^{\sigma}{}_\tau\big)
    \bigg)
  \end{split}
\end{equation}
where we have included the Chern-Simons term as it is also invariant.

\section{4D metricless torsion action}
Following the precepts we have already learnt, we will start by defining a decomposition for the full connection field 
\begin{align}
  \Gamma_{\rho}{}^{\mu}{}_\sigma=\hat\Gamma_{\rho}{}^{\mu}{}_\sigma+T_{\rho}{}^{\mu}{}_\sigma=\hat\Gamma_{\rho}{}^{\mu}{}_\sigma+\epsilon_{\rho\sigma\lambda\kappa}T^{\mu,\lambda\kappa}+A_{[\rho}\delta^\mu_{\nu]}
\end{align}
where we have defined a symmetric connection $\hat\Gamma_{\rho}{}^{\mu}{}_\sigma$ and a vector field $A_\mu$ that gives trace to the antisymetric part of the full connection, and we have also defined a (Curtright) field $T^{\mu,\lambda\kappa}$ that is defined through the symmetry of its indices, it is antisymmetric in the second two indices and it has a cyclic property $T^{\mu,\lambda\kappa}+T^{\lambda,\kappa\mu}+T^{\kappa,\mu\lambda}=0$, in other words that $T^{[\mu,\lambda]\kappa}=\frac{1}{2}T^{\kappa,\lambda\mu}$ just as it happens with the Riemmann tensor $\hat R_{\mu[\nu}{}^\alpha{}_{\lambda]}=\frac{1}{2}\hat R_{\lambda\nu}{}^\alpha{}_{\mu}$. Notice that due to its symmetries $\epsilon_{\rho\sigma\lambda\kappa}T^{\mu,\lambda\kappa}$ is traceless. 

We can now write all the combinations of fields that would presumably be renormalizable with these three independent fields
\begin{equation}
  \begin{split}
    \label{3+1full}
     S[\hat\Gamma_{\mu}{}^{\nu}{}_\rho,T^{\lambda,\mu\nu},A_\mu] =\int\md^4x & \Bigg( A_1 \hat R_{\mu\nu}{}^{\alpha}{}_\beta \hat R_{\rho\sigma}{}^{\beta}{}_\alpha\epsilon^{\mu\nu\rho\sigma}+A_2 \hat R_{\mu\nu}{}^{\alpha}{}_\alpha \hat R_{\rho\sigma}{}^{\beta}{}_\beta\epsilon^{\mu\nu\rho\sigma}
    \\
    & +\Big(B_1 \hat R_{\mu\nu}{}^{\mu}{}_\rho T^{\nu,\alpha\beta}T^{\rho,\gamma\delta}+B_2 \hat R_{\mu\nu}{}^{\alpha}{}_\rho T^{\beta,\mu\nu}T^{\rho,\gamma\delta}+B_3\hat R_{\mu\nu}{}^{\alpha}{}_\rho T^{\mu,\nu\beta}T^{\rho,\gamma\delta}\Big)\epsilon_{\alpha\beta\gamma\delta}
    \\
    & +B_4\hat R_{\mu\nu}{}^{\alpha}{}_\rho T^{\rho,\mu\nu}A_\alpha+B_5\hat R_{\mu\nu}{}^{\rho}{}_\rho T^{\alpha,\mu\nu}A_\alpha+B_6 \hat R_{\mu\nu}{}^{\mu}{}_\rho T^{\rho,\nu\alpha}A_\alpha 
    \\
    & +C_1\hat R_{\mu\nu}{}^{\alpha}{}_\rho \hat\nabla_\alpha T^{\rho,\mu\nu}+C_2\hat R_{\mu\nu}{}^{\rho}{}_\rho \hat\nabla_\alpha T^{\alpha,\mu\nu} +C_3 \hat R_{\mu\nu}{}^{\mu}{}_\rho \hat\nabla_\alpha T^{\rho,\nu\alpha}
    \\
    & +C_4\hat R_{\mu\nu}{}^{\tau}{}_\tau \partial_\rho A_\sigma \epsilon^{\mu\nu\rho\sigma}+D_1T^{\alpha,\mu\nu}T^{\beta,\rho\sigma}\hat\nabla_\gamma T^{\lambda,\gamma\kappa}\epsilon_{\mu\nu\beta\lambda}\epsilon_{\alpha\rho\sigma\kappa}
    \\  
    & + D_2T^{\alpha,\mu\nu}T^{\lambda,\beta\gamma}\hat\nabla_\lambda T^{\delta,\rho\sigma}\epsilon_{\alpha\beta\gamma\delta}\epsilon_{\mu\nu\rho\sigma}+D_3T^{\mu,\alpha\beta}T^{\lambda,\nu\gamma}\hat\nabla_\lambda T^{\delta,\rho\sigma}\epsilon_{\alpha\beta\gamma\delta}\epsilon_{\mu\nu\rho\sigma}
    \\
    & + D_4T^{\lambda,\mu\nu}T^{\kappa,\rho\sigma}\hat\nabla_\lambda A_\kappa \epsilon_{\mu\nu\rho\sigma}+D_5T^{\lambda,\mu\nu}A_\lambda\hat\nabla_\mu A_\nu +D_6T^{\lambda,\mu\nu}A_\mu\hat\nabla_\lambda A_\nu
    \\
    & +D_7A_\lambda  A_\mu\hat\nabla_\nu T^{\lambda,\mu\nu} +E_1\hat\nabla_\alpha T^{\alpha,\mu\nu}\hat\nabla_\beta T^{\beta,\lambda\rho}\epsilon_{\mu\nu\lambda\rho}+E_2\hat\nabla_\beta T^{\alpha,\mu\nu}\hat\nabla_\alpha T^{\beta,\lambda\rho}\epsilon_{\mu\nu\lambda\rho}
    \\
    & +E_3\hat\nabla_\lambda T^{\lambda,\mu\nu}\hat\nabla_\mu A_\nu+E_4\hat\nabla_\mu T^{\lambda,\mu\nu}\hat\nabla_\nu A_\lambda + E_5\partial_\alpha A_\beta\partial_\gamma A_\delta \epsilon^{\alpha\beta\gamma\delta}
    \\
    & +T^{\alpha,\beta\gamma}T^{\delta,\eta\kappa}T^{\lambda,\mu\nu}T^{\rho,\sigma\tau}\Big(\Lambda_1\epsilon_{\beta\gamma\eta\kappa}\epsilon_{\alpha\rho\mu\nu}\epsilon_{\delta\lambda\sigma\tau}+\Lambda_2\epsilon_{\beta\lambda\eta\kappa}\epsilon_{\gamma\rho\mu\nu}\epsilon_{\alpha\delta\sigma\tau}\Big)
    \\
    & +\Lambda_3 T^{\rho,\alpha\beta}T^{\gamma,\mu\nu}T^{\lambda,\sigma\tau}A_\tau \epsilon_{\alpha\beta\gamma\lambda}\epsilon_{\mu\nu\rho\sigma}+\Lambda_4T^{\eta,\alpha\beta}T^{\kappa,\gamma\delta}A_\eta A_\kappa\epsilon_{\alpha\beta\gamma\delta}\Bigg)
  \end{split}
\end{equation}


\subsection*{3+1 decomposition of $T^{\mu,\nu\rho}$ and degrees of freedom}

The  analysis of the degrees of freedom of the 2+1 action is pretty straightforward because it clearly corresponds to a gravitational action plus some extra stuff. In 3+1 dimensions we do not recognize the equivalence of \eqref{3+1full} to anything known, therefore we have to do a proper analysis, for which we are going to propose a decomposition of the fields $A_\mu=\delta_\mu^0 A_0+\delta_\mu^iA_i$ and 
\begin{align}
  T^{\mu,\nu\rho}=\delta^{\mu}_i\delta^{\nu\rho}_{jk}T^{i,jk}+(\delta^{\mu}_0\delta^{\nu\rho}_{ij}-\delta^{\mu}_i\delta^{\nu\rho}_{j0})a^{[ij]}+\delta^{\mu}_i\delta^{\nu\rho}_{j0}T^{(ij)}+\delta^{\mu}_0\delta^{\nu\rho}_{i0}c^i,
\end{align}
where $\delta^{\mu\nu}_{\lambda\kappa}=\delta^{\mu}_{\lambda}\delta^{\nu}_{\kappa}-\delta^{\mu}_{\kappa}\delta^{\nu}_{\lambda}$, $T^{i,jk}\epsilon_{ijk}=0$, $a^{ij}$ is antisymmetric, $T^{ij}$ is a symmetric tensor and $c^i$ an arbitrary vector. In order to make perturbation theory we will expand around an isotropic and homogeneous solution, as these are characteristic of the observable universe. Thus, in order to find such solution we can substitute $A_\mu=\delta_\mu^0 A_0$ and $T^{\mu,\nu\rho}=\delta^{\mu}_i\delta^{\nu\rho}_{j0}T^{ij}$; for a spacially flat metric type, with $k=0$, we would choose $T^{ij}=\delta^{ij}T(t)$. For this particular choice we wish to remark that the contraction of torsion terms
\begin{align}
  T_{\mu}{}^{\sigma}{}_\rho T_{\nu}{}^{\rho}{}_\sigma &= (\epsilon_{\mu\rho\lambda\kappa}T^{\sigma,\lambda\kappa}+A_{[\mu}\delta^\sigma_{\rho]})(\epsilon_{\nu\sigma\delta\eta}T^{\rho,\delta\eta}+A_{[\nu}\delta^\rho_{\sigma]}) \notag \\
  &= \delta^0_\mu\delta^0_\nu\frac{3}{4} A_0^2-24\det(T^{ij})\delta^i_\mu\delta^j_\nu T_{ij},
\end{align}
where $T_{ij}$ is defined as the inverse of $T^{ij}$. Being a covariant tensor, $T_{\mu}{}^{\sigma}{}_\rho T_{\nu}{}^{\rho}{}_\sigma$ can play the role of the metric of the {\it spacetime}, defined as a constructed field $g_{\mu\nu}:= T_{\mu}{}^{\sigma}{}_\rho T_{\nu}{}^{\rho}{}_\sigma $. Similarly, we can define another metric-like tensor, 
\begin{align}
  \tilde g^{\lambda\kappa}:= T_{\mu}{}^{\lambda}{}_\nu T_{\rho}{}^{\kappa}{}_\sigma\epsilon^{\mu\nu\rho\sigma}=-8\delta^\lambda_i\delta^\kappa_jA_0T^{ij},
\end{align}
but we can see that this would not be a four dimensional metric but only the space part of it.

Anyhow, we can now do perturbation theory expanding up to second order the fields around the solution $T^{\mu,\nu\lambda} = \delta^{\mu}_i \delta_{i0}^{\nu\lambda}T + t^{\mu,\nu\lambda}$ and $A_\mu=\delta_\mu^0A_0+a_\mu$ and in the simplest case, where $A_0$ and $T$ are constants in time we can also set $\hat\Gamma_\mu{}^\lambda{}_\nu=\gamma_\mu{}^\lambda{}_\nu$, where the lowcase fields are small perturbations. The expansion is required to be performed around a minimum of the action, therefore we will set linear terms to zero. 
\begin{align}
  \label{3+1fullPerturbed}
   S_0[\hat\Gamma_{\mu}{}^{\nu}{}_\rho,T^{\lambda,\mu\nu},A_\mu]=\int\md^4x\Bigg(-3!A_0T^3\Lambda_3\Bigg),
\end{align}

\begin{equation}
  \begin{split}
    \label{3+1full1stOrder}
    S_1[\hat\Gamma_{\mu}{}^{\nu}{}_\rho,T^{\lambda,\mu\nu},A_\mu] = \int\md^4x &\Bigg(2(-4B_2+B_3)T^2\partial_{[i}\gamma_{0]}{}^k{}_j\epsilon_{0ijk} \\
    & + 2 A_0 T \big((B_4-B_6)\partial_{[i}\gamma_{0]}{}^0{}_i+B_6\partial_{[i}\gamma_{j]}{}^j{}_i\big)\\
    & + 8 T^3 \big(-3D_1\gamma_0{}^0{}_0+(-2D_1+D_2+D_3)\gamma_i{}^i{}_0\big)\\
    & - T A_0^2(D_6+D_7)\gamma_i{}^0{}_i -12\Lambda_3( 2t^{k,k0}+a_0)\Bigg).
  \end{split}
\end{equation}
Therefore, we must set $B_3=4B_2$, $B_4=0=B_6$, $D_1=0$, $D_3=-D_2$,  $D_7=-D_6$ and $\Lambda_3=0$ in order to be sure that the homogeneous and isotropic field configuration minimizes the action. 

The second order perturbative expansion  is

\section*{FALTA}

\begin{itemize}
\item tal vez $a^{ij}$ y $c^i$ sean campo el\'ectrico y magn\'etico

\item study  whether the spacetime solution can arise as a symmetry breaking from thermal fluctuations around $T=0$

\item estudiar la posibilidad de obtener Ho\v{r}ava-Lifshitz gravity en algun limite

\item hacer teoria de perturbaciones alrededor de la solucion constante

\item calcular el limite Newtoniano

\item I am studying how to do the d.o.f. analysis.
\item check the most general action in 4d.
\end{itemize}

\nocite{Tucker:1996sx,Horava:2009uw,Lu:2009em,Gibbs:1995gj,WheelerPre}

\bibliographystyle{aipauth4-1}
\bibliography{References.bib}

\end{document}

%% In the case that all $D_\pi=0$ we are left with an action that depends 
%% directly on a combination of the Riemann and Torsion tensor. 
%% Determination of the equations of motion is straightforward, and we obtain 
%% a combination of $\nabla_\mu T_{\nu}{}^\sigma{}_{\lambda}$ and 
%% $R_{\mu\nu}{}^\sigma{}_{\lambda}$. However, due to the complicated 
%% combination of invariant tensors, we can solve these equations by setting
%%  $\nabla_\mu T_{\nu}{}^\sigma{}_{\lambda}=0$ and 
%% $R_{\mu\nu}{}^\sigma{}_{\lambda}=0$. 

%% Thus, it seems appropriate to decompose it in terms of $
%% \Gamma_{\nu}{}^\lambda{}_{\rho}=\hat\Gamma_{\nu}{}^\lambda{}_{\rho}+\hat T_{\nu}{}^\lambda{}_{\rho}+A_{[\nu}\delta^\lambda_{\rho]}$. Then,
%% \ba
%% R_{\mu\nu}{}^\lambda{}_\rho&=&2\hat\nabla_{[\mu}\hat\Gamma_{\nu]}{}^\lambda{}_{\rho}+2\hat\nabla_{[\mu}\hat T_{\nu]}{}^\lambda{}_{\rho}-2\hat T_{\sigma}{}^\lambda{}_{[\mu}\hat T_{\nu]}{}^\sigma{}_{\rho}+\\ \nonumber&&+\hat\nabla_{[\mu}A_{\nu]}\delta^\lambda_\rho+\delta^\lambda_{[\mu}\hat\nabla_{\nu]}A_\rho-\hat T_\mu{}^\lambda{}_{\nu}A_\rho-\delta^\lambda_{[\mu}\hat T_{\nu]}{}^\sigma{}_{\rho}A_\sigma +\frac{1}{2}\delta^\lambda_{[\mu}A_{\nu]}A_\rho
%% \ea

%% {\bf esto se ve tan feo que creo que es preferible trabajar con la torsion sin hacer la descomposicion de la torsion en $\hat T +A$}

%% Thus, we decompose the connection in terms of its symmetric and antisymmetric parts $
%% \Gamma_{\nu}{}^\lambda{}_{\rho}=\hat\Gamma_{\nu}{}^\lambda{}_{\rho}+\frac{1}{2} T_{\nu}{}^\lambda{}_{\rho}$
%% \be
%% R_{\mu\nu}{}^\lambda{}_\
